% \begin{Date}

\def\twodigits#1{\ifnum#1<10 0\fi\the#1}
\def\datetwo{\leavevmode\hbox{\the\year\twodigits\month\twodigits\day}}

% \end{Date}

% \begin{Packages}

\usepackage[a4paper,left=3cm,right=3cm,top=2.5cm,bottom=2.5cm]{geometry}
\usepackage[headsepline]{scrlayer-scrpage}
\usepackage{mdframed}
\usepackage{chngcntr}
\usepackage{enumitem}
\usepackage{amsmath}
\usepackage{amssymb}
\usepackage{amsthm}
\usepackage{empheq}
\usepackage{float}
\usepackage{thmtools}
\usepackage{mathtools}
\usepackage[svgnames,dvipsnames,table]{xcolor}
\usepackage{adjustbox}
\usepackage{fancybox}
\usepackage{tikz}
\usepackage{tkz-euclide}
\usepackage{pgfplots}
\usepackage[colorlinks=true,pdfusetitle]{hyperref}

% \end{Packages}

% \begin{Package Configuration}

% Hyperref configuration
\hypersetup{pdfkeywords={The Singapore Mathematical Olympiad (SMO) Compendium},pdfsubject={SMO Compendium}}
\hypersetup{urlcolor=RubineRed,linkcolor=RoyalBlue,citecolor=ForestGreen}

% TOC configuration
\AfterTOCHead{\pagestyle{plain}}
\AfterStartingTOC{\clearpage}

% PGFPlots configuration
\pgfplotsset{compat=1.18}
\usepgfplotslibrary{polar, fillbetween}

% Tikz configuration
\usetikzlibrary{arrows, intersections, angles, quotes, calc, decorations.markings, patterns}

% \end{Package Configuration}

% \begin{Tikz Custom Styles}

% Line with arrow midway
\tikzset{->-/.style={decoration={
  markings,
  mark=at position #1 with {\arrow{>}}},postaction={decorate}}}

% \end{Tikz Custom Styles}

% \begin{Solutions, Cases and Subcases}

% -- SOLUTIONS --

% Vanilla solution
\NewDocumentCommand{\sol}{g o}{\noindent\textbf{Solution #1.} \IfNoValueTF{#2}{}{\textit{(#2)}}}

% Solution with credit (typewriter)
\newcommand{\solctt}[2]{
\noindent\textbf{Solution by \href{#1}{\texttt{#2}}.}
}

% Solution with credit (italic)
\newcommand{\solcit}[2]{
\noindent\textbf{Solution by \href{#1}{\textit{#2}}.}
}

% -- CASES --

\NewDocumentCommand{\case}{g o}{\textit{Case #1}\IfNoValueTF{#2}{.}{: \textit{#2}.}}

% -- SUBCASES

\NewDocumentCommand{\subcase}{g o}{\textit{Subcase #1}\IfNoValueTF{#2}{.}{: \textit{#2}.}}

% \end{Solutions, Cases and Subcases}

% \begin{Theorem Environments}

\theoremstyle{plain}
\newtheorem{theorem}{Theorem}
\newtheorem{propcount}{Count}

\newtheorem{claiminner}{Claim}
\newenvironment{claim}[1]{%
  \IfBlankTF{#1}
    {\renewcommand{\theclaiminner}{\unskip}}
    {\renewcommand\theclaiminner{#1}}%
  \claiminner
}{\endclaiminner}

\theoremstyle{definition}
\newtheorem*{definition}{Definition}

\theoremstyle{remark}
\newtheorem*{remark}{Remark}

% \end{Theorem Environments}

% \begin{Question Callout}

\mdfdefinestyle{mdbluebox}{%
    roundcorner=10pt,
    linewidth=1pt,
    skipabove=12pt,
    innerbottommargin=9pt,
    skipbelow=2pt,
    linecolor=RoyalBlue,
    nobreak=true,
    backgroundcolor=TealBlue!5,
}
\declaretheoremstyle[
    headfont=\sffamily\bfseries\color{MidnightBlue},
    mdframed={style=mdbluebox},
    headpunct={\\[3pt]},
    postheadspace={10pt},
    notebraces={\hfill[Ans: }{]},
]{thmbluebox}

% Question environment
\declaretheorem[style=thmbluebox,name=Question,sibling=theorem]{question}
\counterwithin*{question}{subsection}

% \end{Question Callout}

% \begin{Resource Boxes}

% Colours a hyperlink
\newcommand{\hrefcolor}[3][black]{\href{#2}{\color{#1}{#3}}}%

% Creates a resource for a paper (e.g. reviews by Way Tan)
\newcommand{\resp}[1]{
\noindent
\adjustbox{right}{
\fcolorbox{red}{red!20}{\centering \sffamily #1}
}}

% \resp with a coloured hyperlink
\newcommand{\resph}[3][Red]{\resp{\hrefcolor[#1]{#2}{#3}}}

% Creates a resource for a question (e.g. AoPS threads)
\newcommand{\resq}[1]{
\noindent
\adjustbox{right}{
\fcolorbox{ForestGreen}{ForestGreen!5}{\centering \sffamily #1}
}\medbreak}

% \resq with a coloured hyperlink
\newcommand{\resqh}[3][ForestGreen]{\resq{\hrefcolor[#1]{#2}{#3}}}

% \end{Resource Boxes}

% \begin{Enum Links}
\makeatletter
\newcommand{\hyperrefitem}[1][]{%
\stepcounter{\@enumctr}%
\item[{\hyperref[#1]{\csname label\@enumctr\endcsname}}]
}
\makeatother
% \end{Enum Links}

% \begin{Math Commands}
% Most commands taken from echw.sty

% -- LINEAR ALGEBRA --

% Boldface Vectors
\renewcommand{\vec}[1]{\mathbf{#1}}

% Boldface Matrix
\newcommand{\matr}[1]{\bm{#1}}

% 2D Column Vectors
\newcommand*{\cvecii}[2]{\begin{pmatrix} #1 \\ #2 \end{pmatrix}}

% 3D Column Vectors
\newcommand*{\cveciii}[3]{\begin{pmatrix} #1 \\ #2 \\ #3 \end{pmatrix}}

% Dot Product
\def\dotp{\cdot}

% Cross Product
\def\crossp{\times}

% -- GEOMETRY --

% Degree Symbol
\renewcommand{\deg}{^{\circ}}

% Area Operator
\DeclareMathOperator{\area}{Area}

% Volume Operator
\DeclareMathOperator{\volume}{Volume}

% Length Operator
\DeclareMathOperator{\length}{Length}

% -- NUMBER THEORY --

\newcommand{\legendre}[2]{\bp{\frac{#1}{#2}}}

% -- COMBINATORICS --

% Permute
\newcommand\perm[2]{\prescript{#1\mkern-2.5mu}{}P_{#2}}

% Choose
\newcommand\comb[2]{\prescript{#1\mkern-0.5mu}{}C_{#2}}

% -- COMPLEX NUMBERS --

% Complex Conjugate
\def\cconj{^\ast}

% Real part
\RenewDocumentCommand{\Re}{g}{%
\IfNoValueTF{#1}{\operatorname{Re}}%
{\operatorname{Re}\of{#1}}}

% Imaginary part
\RenewDocumentCommand{\Im}{g}{%
\IfNoValueTF{#1}{\operatorname{Im}}%
{\operatorname{Im}\of{#1}}}

% -- MISC --

% LHS
\DeclareMathOperator{\LHS}{LHS}

% RHS
\DeclareMathOperator{\RHS}{RHS}

% -- BRACKETS --

\newcommand{\bc}[1]{\left\lbrace #1 \right\rbrace} 
\newcommand{\bp}[1]{\left( #1 \right)} 
\newcommand{\bs}[1]{\left[ #1 \right]} 
\newcommand{\ba}[1]{\langle #1 \rangle}
\newcommand{\abs}[1]{\left\lvert #1 \right\rvert}
\newcommand{\absx}[1]{\lvert #1 \rvert}
\newcommand{\norm}[1]{\left\lVert #1 \right\rVert}
\newcommand{\normx}[1]{\lVert #1 \rVert}
\newcommand{\floor}[1]{\left\lfloor #1 \right\rfloor}
\newcommand{\ceil}[1]{\left\lceil #1 \right\rceil}
\newcommand{\floorx}[1]{\lfloor #1 \rfloor}
\newcommand{\ceilx}[1]{\lceil #1 \rceil}
\newcommand{\of}[1]{{\left( #1 \right)}}

% -- OPERATORS --

% LCM
\DeclareMathOperator{\oldlcm}{lcm}
\NewDocumentCommand{\lcm}{g}{
    \IfNoValueTF{#1}
        {\oldlcm}
        {{\oldlcm}{\bp{#1}}}
    }

% GCD
\let\oldgcd\gcd
\RenewDocumentCommand{\gcd}{g}{
\IfNoValueTF{#1}
    {\oldgcd}
    {{\oldgcd}{\bp{#1}}}
}

% Sine
\let\oldsin\sin
\RenewDocumentCommand{\sin}{o g}{%
\IfNoValueTF{#2}{\oldsin}%
{\,{\oldsin\IfValueT{#1}{^{#1}}}{\bp{#2}}}}

% Cosine
\let\oldcos\cos
\RenewDocumentCommand{\cos}{o g}{%
\IfNoValueTF{#2}{\oldcos}%
{\,{\oldcos\IfValueT{#1}{^{#1}}}{\bp{#2}}}}

% Tangent
\let\oldtan\tan
\RenewDocumentCommand{\tan}{o g}{%
\IfNoValueTF{#2}{\oldtan}%
{\,{\oldtan\IfValueT{#1}{^{#1}}}{\bp{#2}}}}

% Cosecant
\let\oldcsc\csc
\RenewDocumentCommand{\csc}{o g}{%
\IfNoValueTF{#2}{\oldcsc}%
{\,{\oldcsc\IfValueT{#1}{^{#1}}}{\bp{#2}}}}

% Secant
\let\oldsec\sec
\RenewDocumentCommand{\sec}{o g}{%
\IfNoValueTF{#2}{\oldsec}%
{\,{\oldsec\IfValueT{#1}{^{#1}}}{\bp{#2}}}}

% Cotangent
\let\oldcot\cot
\RenewDocumentCommand{\cot}{o g}{%
\IfNoValueTF{#2}{\oldcot}%
{\,{\oldcot\IfValueT{#1}{^{#1}}}{\of{#2}}}}

% Arcsine
\let\oldarcsin\arcsin
\RenewDocumentCommand{\arcsin}{o g}{%
\IfNoValueTF{#2}{\oldarcsin}%
{\,{\oldarcsin\IfValueT{#1}{^{#1}}}{\bp{#2}}}}

% Arccosine
\let\oldarccos\arccos
\RenewDocumentCommand{\arccos}{o g}{%
\IfNoValueTF{#2}{\oldarccos}%
{\,{\oldarccos\IfValueT{#1}{^{#1}}}{\bp{#2}}}}

% Arctangent
\let\oldarctan\arctan
\RenewDocumentCommand{\arctan}{o g}{%
\IfNoValueTF{#2}{\oldarctan}%
{\,{\oldarctan\IfValueT{#1}{^{#1}}}{\bp{#2}}}}

% Logarithm
\let\oldln\ln
\RenewDocumentCommand{\ln}{g}{%
\IfNoValueTF{#1}{\oldln}%
{\,{\oldln}{\bp{#1}}}}

% Exponential
\let\oldexp\exp
\RenewDocumentCommand{\exp}{g}{%
\IfNoValueTF{#1}{\oldexp}%
{\,{\oldexp}{\bp{#1}}}}

% Argument
\let\oldarg\arg
\RenewDocumentCommand{\arg}{g}{%
\IfNoValueTF{#1}{\oldarg}%
{\,{\oldarg}{\bp{#1}}}}

% Indicator
\NewDocumentCommand{\ind}{g}{%
\IfNoValueTF{#1}{\operatorname{\mathbb{1}}}%
{\,\mathbb{1}_{#1}}}

% Trace
\NewDocumentCommand{\tr}{g}{%
\IfNoValueTF{#1}{\operatorname{tr}}%
{\,{\operatorname{tr}}{\bp{#1}}}}

% Determinant
\let\olddet\det
\RenewDocumentCommand{\det}{g}{%
\IfNoValueTF{#1}{\olddet}%
{\,{\olddet}{\bp{#1}}}}

% Variance
\NewDocumentCommand{\var}{o g}{%
\IfNoValueTF{#2}{\operatorname{var}\IfValueT{#1}{_{#1}}}%
{\,{\operatorname{var}\IfValueT{#1}{_{#1}}}{\bp{#2}}}}

% Covariance
\NewDocumentCommand{\cov}{o g}{%
\IfNoValueTF{#2}{\operatorname{cov}\IfValueT{#1}{_{#1}}}%
{\,{\operatorname{cov}\IfValueT{#1}{_{#1}}}{\bp{#2}}}}

% Correlation
\NewDocumentCommand{\corr}{o g}{%
\IfNoValueTF{#2}{\operatorname{corr}\IfValueT{#1}{_{#1}}}%
{\,{\operatorname{corr}\IfValueT{#1}{_{#1}}}{\bp{#2}}}}

% Standard deviation
\NewDocumentCommand{\sd}{o g}{%
\IfNoValueTF{#2}{\operatorname{sd}\IfValueT{#1}{_{#1}}}%
{\,{\operatorname{sd}\IfValueT{#1}{_{#1}}}{\bp{#2}}}}

% Standard error
\NewDocumentCommand{\se}{o g}{%
\IfNoValueTF{#2}{\operatorname{se}\IfValueT{#1}{_{#1}}}%
{\,{\operatorname{se}\IfValueT{#1}{_{#1}}}{\bp{#2}}}}

% Maximum
\let\oldmax\max
\RenewDocumentCommand{\max}{o g}{%
\IfNoValueTF{#2}{\oldmax\IfValueT{#1}{_{#1}}}%
{\,{\oldmax\IfValueT{#1}{_{#1}}}{\bc{#2}}}}

% Supremum
\let\oldsup\sup
\RenewDocumentCommand{\sup}{o g}{%
\IfNoValueTF{#2}{\oldsup\IfValueT{#1}{_{#1}}}%
{\,{\oldsup\IfValueT{#1}{_{#1}}}{\bc{#2}}}}

% Argmax
\DeclareMathOperator*{\argmax}{argmax}

% Minimum
\let\oldmin\min
\RenewDocumentCommand{\min}{o g}{%
\IfNoValueTF{#2}{\oldmin\IfValueT{#1}{_{#1}}}%
{\,{\oldmin\IfValueT{#1}{_{#1}}}{\bc{#2}}}}

% Infimum
\let\oldinf\inf
\RenewDocumentCommand{\inf}{o g}{%
\IfNoValueTF{#2}{\oldinf\IfValueT{#1}{_{#1}}}%
{\,{\oldinf\IfValueT{#1}{_{#1}}}{\bc{#2}}}}

% Argmin
\DeclareMathOperator*{\argmin}{argmin}

% Expectation
\NewDocumentCommand{\E}{o g}{%
\IfNoValueTF{#2}{\operatorname{\mathds{E}}\IfValueT{#1}{_{#1}}}%
{{\operatorname{\mathds{E}}\IfValueT{#1}{_{#1}}}{\bp{#2}}}}

% Probability
\RenewDocumentCommand{\P}{o g g}{%
\IfNoValueTF{#2}{\operatorname{\mathds{P}}\IfValueT{#1}{_{#1}}}%
{{\operatorname{\mathds{P}}\IfValueT{#1}{_{#1}}}{\bp{#2\IfValueT{#3}{\mid #3}}}}}

% -- CALCULUS --

% Differential
\renewcommand{\d}[1]{\mathop{\mathrm{d}#1}}

% Ordinary derivative in display
\NewDocumentCommand{\der}{o g g}{%
\dfrac{\mathrm{d}\IfValueT{#1}{^{#1}} #2}{\mathrm{d} #3\IfValueT{#1}{^{#1}}}}

% Ordinary derivative in text
\NewDocumentCommand{\derx}{o g g}{%
\mathrm{d}\IfValueT{#1}{^{#1}} #2/\mathrm{d} #3\IfValueT{#1}{^{#1}}}

% Partial derivative in display
\NewDocumentCommand{\pder}{o g g g}{%
\IfNoValueTF{#4}{\dfrac{\partial\IfValueT{#1}{^{#1}} #2}{\partial #3\IfValueT{#1}{^{#1}}}}%
{\left.\dfrac{\partial\IfValueT{#1}{^{#1}} #2}{\partial #3\IfValueT{#1}{^{#1}}}\right\vert_{#4}}}

% Partial derivative in text
\NewDocumentCommand{\pderx}{o g g g}{%
\IfNoValueTF{#4}{\partial\IfValueT{#1}{^{#1}} #2/\partial #3\IfValueT{#1}{^{#1}}}%
{\left.\partial\IfValueT{#1}{^{#1}} #2/\partial #3\IfValueT{#1}{^{#1}}\right\vert_{#4}}}

% -- BLACKBOARD LETTERS --

\def\RR{\mathbb{R}}
\def\NN{\mathbb{N}}
\def\ZZ{\mathbb{Z}}
\def\QQ{\mathbb{Q}}
\def\CC{\mathbb{C}}
\def\FF{\mathbb{F}}

% -- CALIGRAPHIC LETTERS --

\def\Ac{\mathcal{A}}
\def\Bc{\mathcal{B}}
\def\Cc{\mathcal{C}}
\def\Dc{\mathcal{D}}
\def\Ec{\mathcal{E}}
\def\Fc{\mathcal{F}}
\def\Gc{\mathcal{G}}
\def\Hc{\mathcal{H}}
\def\Ic{\mathcal{I}}
\def\Jc{\mathcal{J}}
\def\Kc{\mathcal{K}}
\def\Lc{\mathcal{L}}
\def\Mc{\mathcal{M}}
\def\Nc{\mathcal{N}}
\def\Oc{\mathcal{O}}
\def\Pc{\mathcal{P}}
\def\Qc{\mathcal{Q}}
\def\Rc{\mathcal{R}}
\def\Sc{\mathcal{S}}
\def\Tc{\mathcal{T}}
\def\Uc{\mathcal{U}}
\def\Vc{\mathcal{V}}
\def\Wc{\mathcal{W}}
\def\Xc{\mathcal{X}}
\def\Yc{\mathcal{Y}}
\def\Zc{\mathcal{Z}}

% -- GREEK LETTERS --

\def\a{\alpha}
\def\b{\beta}
\def\c{\chi}
\def\de{\delta}
\def\D{\Delta}
\def\e{\epsilon}
\def\ve{\varepsilon}
\def\f{\phi}
\def\vf{\varphi}
\def\F{\Phi}
\def\g{\gamma}
\def\G{\Gamma}
\def\h{\eta}
\def\i{\iota}
\def\k{\kappa}
\def\vk{\varkappa}
\def\l{\lambda}
\def\L{\Lambda}
\def\m{\mu}
\def\n{\nu}
\def\o{\omega}
\def\O{\Omega}
\def\p{\psi}
\def\r{\rho}
\def\vr{\varrho}
\def\s{\sigma}
\def\vs{\varsigma}
\def\S{\Sigma}
\def\t{\theta}
\def\vt{\vartheta}
\def\T{\Theta}
\def\u{\upsilon}
\def\U{\Upsilon}
\def\x{\xi}
\def\X{\Xi}
\def\z{\zeta}

% -- ACCENTS --

\newcommand{\ol}[1]{\overline{#1}}
\newcommand{\oa}[1]{\overrightarrow{#1}}
\newcommand{\ul}[1]{\underline{#1}}
\newcommand{\wh}[1]{\widehat{#1}}
\newcommand{\wt}[1]{\widetilde{#1}}

%\end{Math Commands}

%\begin{Headers}

\clearpairofpagestyles
\ohead*{\pagemark}
\chead{\headmark}

%\end{Headers}
