\subsection{Round 2 Problems}

Solutions can be found in Section~\ref{S::2022-O-2}.

\begin{enumerate}
    \hyperrefitem[A::2022-O-2-1] For $\triangle ABC$ and its circumcircle $\o$, draw the tangents at $B$, $C$ to $\o$ meeting at $D$. let the line $AD$ meet the circle with centre $D$ and radius $DB$ at $E$ inside $\triangle ABC$. Let $F$ be the point on the extension of $EB$ and $G$ be the point on the segment $EC$ such that $\angle AFB = \angle AGE = \angle A$. Prove that the tangent at $A$ to the circumcircle of $\triangle AFG$ is parallel to $BC$.
    \hyperrefitem[A::2022-O-2-2] Prove that if the length and breadth of a rectangle are both odd integers, then there does not exist a point $P$ inside the rectangle such that each of the distances from $P$ to the 4 corners of the rectangle is an integer.
    \hyperrefitem[A::2022-O-2-3] Find all functions $f : \ZZ^+ \to \ZZ^+$ satisfying \[m!! + n!! \mid f(m)!! + f(n)!!\] for each $m, n \in \ZZ^+$, where $n!! = (n!)!$ for all $n \in \ZZ^+$.
    \hyperrefitem[A::2022-O-2-4] Let $n$, $k$, $1 \leq k \leq n$ be fixed integers. Alice has $n$ cards in a row, where the card in position $i$ has the label $i + k$ (or $i + k - n$ if $i + k > n$). Alice starts by colouring each card either red or blue. Afterwards, she is allowed to make several moves, where each move consists of choosing two cards of different colours and swapping them. Find the minimum number of moves she has to make (given that she chooses the colouring optimally) to put the cards in order (i.e. card $i$ is at position $i$).
    \hyperrefitem[A::2022-O-2-5] Let $n \geq 2$ be a positive integer. For any integer $a$, let $P_a(x)$ denote the polynomial $x^n + ax$. Let $p$ be a prime number and define the set $S_a$ as the set of residues mod $p$ that $P_a(x)$ attains. That is, \[S_a = \bc{b \mid 0 \leq b \leq p -1, \text{ and there is $c$ such that $P_a(x) \equiv p \pmod{p}$}}.\] Show that the expression $\frac1{p-1} \sum_{a=1}^{p-1} \abs{S_a}$ is an integer.
\end{enumerate}
