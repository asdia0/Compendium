\subsection{Round 2 Solutions}\label{S::2024-O-2}
  
\resph{https://simoxmenblog.blogspot.com/2024/06/smo-open-2024-speedrun.html}{Review by \textit{Glen Lim}}

\begin{question}\label{A::2024-O-2-1}
    In triangle $ABC$, $\angle B = 90\deg$, $AB > BC$, and $P$ is the point such that $BP = BC$ and $\angle APB = 90\deg$, where $P$ and $C$ lie on the same side of $AB$. Let $Q$ be the point on $AB$ such that $AP = AQ$, and let $M$ be the midpoint of $QC$. Prove that the line through $M$ parallel to $AP$ passes through the midpoint of $AB$.
\end{question}

\resqh{https://artofproblemsolving.com/community/c6h3371488p31382620}{AoPS thread}

\begin{center}
    \begin{tikzpicture}[scale=4]
        \coordinate[label=above left:$A$] (A) at (0, 1.3);
        \coordinate[label=below left:$B$] (B) at (0, 0);
        \coordinate[label=below right:$C$] (C) at (1, 0);
        \coordinate[label=right:$P$] (P) at (0.639, 1/1.3);
        \coordinate[label=left:$Q$] (Q) at (0, 1.3 - 0.831);
        \coordinate[label=below:$M$] (M) at (0.5, 0.235);
        
        \draw (A) -- (B) -- (C) -- (A);
        \draw (P) -- (B);
        \draw (Q) -- (C);

        \draw[dashed] (0, 1.3) arc (90:-90:0.65);
        \draw[dashed] (0, 1) arc(90:0:1);

        \draw pic [draw, angle radius=2mm, ""] {right angle = A--B--C};

        \draw pic [draw, angle radius=2mm, ""] {right angle = A--P--B};

        \fill (0, 0.65) circle[radius=0.5pt];

        \draw[->-=0.5, very thick] (M) -- (0, 0.65);
        \draw[->-=0.5, very thick] (P) -- (A);
    \end{tikzpicture}
\end{center}

We will use coordinate bashing to solve this problem. Firstly, let $A(0, a)$, $B(0, 0)$, and $C(1, 0)$. Note that we fix $C(1, 0)$ as we can always scale the triangle to make $BC = 1$.

Since $BP = BC = 1$, we know that $P$ lies on the unit circle $x^2 + y^2 = 1$. Furthermore, since $\angle APB = 90\deg$, $P$ also lies on the circle with diameter $AB$, which has equation $x^2 + (y - \frac12 a)^2 = (\frac12 a)^2$. Solving the two equations simultaneously, along with the constraint $x > 0$ (since $P$ and $C$ lie on the same side of $AB$), we get $P(\frac1a \sqrt{a^2 - 1}, \frac1a)$.

Now note that $AQ = AP = \sqrt{(\frac1a \sqrt{a^2 - 1})^2 + (\frac1a - a)^2} = \sqrt{a^2 - 1}$. Hence, $Q(0, a - \sqrt{a^2 - 1})$. Thus, $M(\frac12, \frac12 (a - \sqrt{a^2 - 1}))$.

Note that the gradient of $AP$ is $\frac{a-1/a}{0 - \sqrt{a^2 - 1}/a} = -\sqrt{a^2 - 1}$. Let $l$ be the line passing through $M$ parallel to $AP$. By the point-slope formula, $l$ has the equation \[l : y - \frac{a-\sqrt{a^2 - 1}}{2} = -\sqrt{a^2 - 1} \bp{x - \frac12}.\] When $x = 0$, we have $y = \frac{a}2$. Hence, $l$ passes through $(0, \frac12 a)$, which is the midpoint of $AB$ as desired. \qed

\remark{The condition $AB > BC$ is equivalent to $a > 1$, which is crucial in ensuring that $\sqrt{a^2 - 1}$ remains real.}

\clearpage
\begin{question}[$\frac12 n{(n+1)}$]\label{A::2024-O-2-2}
    Let $n$ be a fixed positive integer. Find the minimum value of \[\frac{x_1^3 + \cdots + x_n^3}{x_1 + \cdots + x_n}\] where $x_1, x_2, \ldots, x_n$ are distinct positive integers.
\end{question}

\resqh{https://artofproblemsolving.com/community/c6h3371489p31382630}{AoPS thread}

\solctt{https://artofproblemsolving.com/community/c6h3371489p31383129}{KHOMNYO2} Without loss of generality, suppose $x_1 < x_2 < \cdots < x_n$. We begin by showing $\sum_{i=1}^{n} x_i^3 \geq (\sum_{i=1}^n x_i)^2$ via induction. The case where $n = 1$ is trivial. Suppose that $\sum_{i=1}^{k} x_i^3 \geq (\sum_{i=1}^k x_i)^2$ for some $k$. Consider $(\sum_{i=1}^{k+1} x_i)^2$:
\begin{align*}
    \bp{\sum_{i=1}^{k+1} x_i}^2 &= \bp{\sum_{i=1}^{k} x_i}^2 + 2x_{k+1}\sum_{i=1}^{k} x_i + x_{k+1}^2\\
    &\leq \sum_{i=1}^{k} x_i^3 + 2x_{k+1}\sum_{i=1}^{k} x_i + x_{k+1}^2\\
    &= \sum_{i=1}^{k+1} x_i^3 + x_{k+1} \bp{2\sum_{i=1}^{k} x_i + x_{k+1} - x_{k+1}^2}.
\end{align*}
It hence suffices to show that $x_{k+1}^2 \geq 2\sum_{i=1}^{k} x_i + x_{k+1}$. Rearranging, we get the equivalent statement $\frac12 x_{k+1} (x_{k+1} - 1) \geq \sum_{i=1}^k x_i$. Indeed, we see that \[\frac{x_{k+1}(x_{k+1} - 1)}2 \geq \frac{x_k (x_k + 1)}2 = 1 + 2 + \cdots + x_k \geq x_1 + x_2 + \cdots + x_k\] where the last inequality holds because $\bc{x_1, x_2, \ldots, x_k} \subseteq \bc{1, 2, \ldots, x_k}$. This closes the induction.

Applying our newly-established inequality to the problem at hand, we see that \[\frac{x_1^3 + \cdots + x_n^3}{x_1 + \cdots + x_n} \geq \frac{\bp{x_1 + \cdots + x_n}^2}{x_1 + \cdots + x_n} \geq x_1 + \cdots + x_n \geq 1 + \cdots + n = \frac{n(n+1)}2.\]

\begin{question}\label{A::2024-O-2-3}
    Prove that for every positive integer $n$ there exists an $n$-digit number divisible by $5^n$ all of whose digits are odd.
\end{question}

\resqh{https://artofproblemsolving.com/community/c6h53661p336189}{AoPS thread}

We prove via induction that for all $n \in \NN$, there must exist some $n$-digit number $a_n$  divisible by $5^n$ all of whose digits are odd. When $n = 1$, it is trivial to see that $a_1 = 5$. Now suppose $a_k$ exists for some $k \in \NN$. Let $S = \bc{1, 3, 5, 7, 9}$. We now show that there exists an $s \in S$ such that $a_{k+1} = s \cdot 10^k + a_k$.

Observe that $s \cdot 10^k + a_k = 5^k \bp{s \cdot 2^k + m}$. Recall that $2^k$ can only end with a 2, 4, 6, or 8 in base 10. Hence, the only residues $2^k$ can take on modulo 5 are 1, 2, 3, and 4. Meanwhile, $s$ can take on any residue modulo 5. It is thus obvious that $s \cdot 2^k$ can take on any residue modulo 5. Hence, by picking $s \equiv -m \pmod{5}$, we will have $5^{k+1} \mid 5^k \bp{s \cdot 2^k + m}$ as desired. This closes the induction. \qed

\begin{question}[$n+1$]\label{A::2024-O-2-4}
    Alice and Bob play a game. Bob starts by picking a set $S$ consisting of $M$ vectors of length $n$ with entries either 0 or 1. Alice picks a sequence of numbers $y_1 \leq y_2 \leq \cdots \leq y_n$ from the interval $[0, 1]$, and a choice of real numbers $x_1, x_2, \ldots, x_n \in \RR$. Bob wins if he can pick a vector $(z_1, z_2, \ldots, z_n) \in S$ such that \[\sum_{i=1}^n x_i y_i \leq \sum_{i=1}^n x_i z_i,\] otherwise Alice wins. Determine the minimum value of $M$ so that Bob can guarantee a win.
\end{question}

\resqh{https://artofproblemsolving.com/community/c6h3371493p31382662}{AoPS thread}

\solctt{https://simoxmenblog.blogspot.com/2024/09/behind-scenes-of-smo-q4-part-1-proposal.html}{DVDthe1st} Note that we can rewrite Bob's winning condition as \[\ba{\vec x, \vec y} \leq \max[\vec z \in S] \ba{\vec x, \vec z}.\] Geometrically, Bob wins if he can find some $\vec z \in S$ whose projection on $\vec x$ is longer than the projection of $\vec y$ on $\vec x$. Hence, $\vec y$ must be in the convex hull of $S$. However, observe that the space of all possible $\vec y$ forms a simplex in $n$-dimensional space. In particular, this simplex has $n+1$ vertices of the form \[\vec v_i = (\underbrace{0, \ldots, 0}_{i}, \underbrace{1, \ldots, 1}_{n-i}).\] Thus, picking $S = \bc{\vec v_0, \vec v_1, \ldots, \vec v_n}$ will guarantee a win for Bob, whence $M = n+1$. We now show that this is indeed the minimum.

Suppose there exists some $\vec v_i \notin S$. Alice can capitalize on this by picking \[\vec x = (\underbrace{-1, \ldots, -1}_{i}, \underbrace{1, \ldots, 1}_{n-i}) \text{ and } \vec y = \vec v_i.\] This is because $\vec x \dotp \vec y$ attains its maximum only when $\vec y = \vec v_i$. Hence, Bob can only win by choosing $\vec z = \vec v_i$. But because $\vec v_i \notin S$, he cannot do so and will thus lose.

\begin{question}[$p+1$]\label{A::2024-O-2-5}
    Let $p$ be a prime number. Determine the largest possible $n$ such that the following holds: it is possible to fill an $n \times n$ table with integers $a_{ik}$ in the $i$th row and $k$th column, for $1 \leq i, k \leq n$, such that for any quadruple $i$, $j$, $k$, $l$ with $1 \leq i < j \leq n$ and $1 \leq k < l \leq n$, the number $a_{ik}a_{jl} - a_{il}a_{jk}$ is not divisible by $p$.
\end{question}

\resqh{https://artofproblemsolving.com/community/c6h3371495p31382678}{AoPS thread}

\solcit{https://simoxmenblog.blogspot.com/2024/06/smo-open-2024-speedrun.html}{Glen Lim} Observe that the condition $p \nmid a_{ik}a_{jl} - a_{il}a_{jk}$ is equivalent to \[
\begin{vmatrix}
    a_{ik} & a_{il}\\
    a_{jk} & a_{jl}
\end{vmatrix} \not\equiv 0 \pmod{p}\] That is, $(a_{ik}, a_{jk})$ and $(a_{il}, a_{jl})$ are linearly independent. $n$ is thus the maximum number of vectors in $\FF_p^2$ that are pairwise linearly independent. 

We now show that $n \leq p + 1$. Take an arbitrary non-zero vector $\vec v \in \FF_p^2$. Then there are clearly $p-1$ non-zero vectors parallel to $\vec v$, namely $\vec v, 2\vec v, \ldots, (p-1)\vec v$. Since there are $p^2 - 1$ non-zero vectors in $\FF_p^2$, the maximum number of pairwise linearly independent vectors is $\frac{p^2 - 1}{p - 1} = p + 1$. Hence, $n \leq p+1$. Indeed, we can construct the following set with $p+1$ vectors that are pairwise linearly independent: \[\bc{\cvecii10, \cvecii01, \cvecii11, \cvecii21, \ldots, \cvecii{p-1}1}.\]

To finish off the problem, we construct a valid $(p + 1) \times (p + 1)$ grid. We do so as follows:
\begin{itemize}
    \item $a_{11} = 0$ (the top left corner is 0)
    \item $a_{1i} = a_{i1} = 1$ for $2 \leq i \leq p+1$ (all other cells in the first row and first column are 1)
    \item $a_{ik} = i-k$ for $2 \leq i, j \leq p + 1$ (cells in the remaining $p \times p$ grid are the difference between the row number and the column number)
\end{itemize}
As an example, the following grid shows the $p = 5$ case.
\begin{table}[h]
    \centering
    \begin{tabular}{|l|l|l|l|l|l|}
    \hline
    0 & 1 & 1 & 1 & 1 & 1 \\ \hline
    1 & 0 & 1 & 2 & 3 & 4 \\ \hline
    1 & 4 & 0 & 1 & 2 & 3 \\ \hline
    1 & 3 & 4 & 0 & 1 & 2 \\ \hline
    1 & 2 & 3 & 4 & 0 & 1 \\ \hline
    1 & 1 & 2 & 3 & 4 & 0 \\ \hline
    \end{tabular}
\end{table}

We now show that $p \nmid a_{ik}a_{jl} - a_{il}a_{jk}$.

\case{1} Consider the case where $i = 1$. Then $a_{1k}a_{jl} - a_{1l}a_{jk} = a_{jl} - a_{jk}$. However, each cell in the $j$th row has a different number, whence it is clear that $a_{jl} - a_{jk} \not\equiv 0 \pmod{p}$. A similar argument works for $k=1$.

\case{2} Consider the case where $i, k \neq 1$. Then $a_{ik}a_{jl} - a_{il}a_{jk} = (i-k)(j-l) - (i-l)(j-k) = (i-j)(k-l)$, which clearly cannot be 0 modulo $p$.

Hence, $\max n = p + 1$.