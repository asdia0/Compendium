\subsection{Round 2 Solutions}\label{S::2024-S-2}

\begin{resources}
    Review by \resit{https://www.youtube.com/watch?v=Mrjm7aWXoNQ}{Way Tan}, \res{https://artofproblemsolving.com/community/c3938466_2024_singapore_senior_math_olympiad}{AoPS threads}
\end{resources}

\begin{question}\label{A::2024-S-2-1}
    In an acute triangle $ABC$, $AC > AB$, $D$ is the point on $BC$ such that $AD = AB$. Let $\o_1$ be the circle through $C$ tangent to $AD$ at $D$, and $\o_2$ the circle through $C$ tangent to $AB$ at $B$. Let $F$ ($\neq C$) be the second intersection of $\o_1$ and $\o_2$. Prove that $F$ lies on $AC$.
\end{question}
\begin{center}
    \begin{tikzpicture}
        \coordinate[label=below left:$A$] (A) at (0, 0);
        \coordinate[label=above left:$B$] (B) at (1.42, 4.48);
        \coordinate[label=below right:$C$] (C) at (8, 0);
        \coordinate[label=above:$D$] (D) at (3.648, 2.963);
        \coordinate[label=below left:$F$] (F) at (2.748, 0);
        \coordinate (O1) at (5.374, 0.821);
        \coordinate (O2) at (5.375, 3.217);
        \coordinate[label=above right:$E$] (D') at (7.959, 6.466);

        \draw (O1) circle[radius=2.7511];
        \draw[dashed] (O2) circle[radius=4.1518];

        \draw (A) -- (B) -- (C) -- (A);
        \draw (A) -- (D);
        \draw[dashed] (D) -- (D');
        \draw (D) -- (F);

        \draw pic [draw, angle radius=8mm, ""] {angle = A--B--D};
        \draw pic [draw, angle radius=8mm, ""] {angle = B--D--A};
        \draw pic [draw, angle radius=8mm, ""] {angle = C--D--D'};
        \draw pic [draw, angle radius=8mm, ""] {angle = C--F--D};

        \draw pic [draw, angle radius=10mm, ""] {angle = A--D--F};
        \draw pic [draw, angle radius=12mm, ""] {angle = A--D--F};

        \draw pic [draw, angle radius=10mm, ""] {angle = D--C--F};
        \draw pic [draw, angle radius=12mm, ""] {angle = D--C--F};

        \node at (7, 3.4) {$\o_1$};
        \node at (5.5, 7.1) {$\o_2$};
        
        \tkzMarkSegment[pos=.5,mark=|](A,B);
        \tkzMarkSegment[pos=.5,mark=|](A,D);
    \end{tikzpicture}
\end{center}
\begin{solution}
    Let $\a = \angle BAD$. Since $\triangle BAD$ is isosceles, we also have $\angle BDA = \a$. Let $E$ be the intersection between $\o_2$ and $AD$ extended. Then $\angle EDC = \a$. By the alternate segment theorem, we get $\angle DFC = \a$. Let $\b = \angle DCF$ by the alternate segment theorem, we also have $\angle ADC = \b$.

    \begin{claim*}
        $BDFA$ is cyclic.
    \end{claim*}
    \begin{proof}
        Observe that $\angle CDF = 180\deg - \a - \b$. Hence, $\angle CAD = 180\deg - \angle ADC - \angle DCA = \a - \b$. Since $\angle BAD = 180\deg - 2\a$, we have \[\angle BAF = (\a - \b) + 180\deg - 2\a = 180\deg - (\a + \b) = 180\deg - \angle BDF.\]
    \end{proof}
    
    Since $BDFA$ is cyclic, we have must have $\angle DFA = 180\deg - \angle DBA = 180\deg - \a$. Thus, $\angle AFD + \angle DFC = 180\deg$, whence $AFC$ is a straight line.
\end{solution}
\begin{solution}[\credit{https://www.youtube.com/watch?v=Mrjm7aWXoNQ}{Way Tan}]
    Note that $F$ and $C$ obviously lie on the radical axis of $\o_1$ and $\o_2$. Observe that the power of $A$ with respect to $\o_1$ and $\o_2$ is $AD^2$ and $AB^2$ respectively. However, because $AD = AB$, the two powers are equal, whence $A$ also lies on the radical axis. Thus, $F$ lies on $AC$. \qed
\end{solution}

\begin{question}[${(-17, -130), (-17, 128), (7, -130), (7, 128)}$]\label{A::2024-S-2-2}
    Find all integer solutions of the equation \[y^2 + 2y = x^4 + 20x^3 + 104x^2 + 40x + 2003.\]
\end{question}
\begin{solution*}
    Completing the square, we have \[(y + 1)^2 - (x^2 + 10x + 2)^2 = 2000,\] which factors as \[(y + x^2 + 10x + 3)(y - x^2 - 10x - 1) = 2000.\] Let $A = y + x^2 + 10x + 3$ and $B = y - x^2 - 10x - 1$. We clearly have \[\frac{A + B}{2} = y + 1 \implies y = \frac{A + B}{2} - 1\] and \[\frac{A - B}{2} = x^2 + 10x + 2 = (x+5)^2 - 23 \implies (x+5)^2 = \frac{A-B}{2} + 23 \tag{1}.\] Going through all possible factor pairs of 2000, we see that only \[(a, b) \in \bc{(-8, -250), (250, 8)}\] gives a perfect square as in (1). Thus, \[(x, y) \in \bc{(-17, -130), (-17, 128), (7, -130), (7, 128)}.\]
\end{solution*}

\begin{question}[11]\label{A::2024-S-2-3}
    Find the smallest positive integer $n$ for which there exist integers $x_1 < x_2 < \cdots < x_n$ such that every integer from 1000 to 2000 can be written as a sum of some of the integers from $x_1, x_2, \ldots, x_n$ without repetition.
\end{question}
\begin{solution*}[\credittt{https://artofproblemsolving.com/community/c6h3344168p30996552}{joeym2011}]
    Observe that there are exactly $2^n$ ways $x_1, \ldots, x_n$ can be summed. Since we want at least 1001 different sums (one for each number between 1000 and 2000), we require $n \geq \ceil{\log_2 1001} = 10$.

    Now observe that we need an $x_i \equiv 1 \pmod{2}$, if not we cannot attain any odd integers. Next, we also need an $x_i \equiv 2 \pmod{4}$, if not we cannot attain any integer congruent to $3 \pmod{4}$. It follows similarly that we require an $x_i \equiv 2^{k-1} \pmod{2^k}$ for $k = 1, \ldots, 11$. Since $k \geq 11$, it follows that $n \geq 11$.

    It is trivial to see that $n = 11$ is possible: simply take $x_i = 2^{i-1}$ and look at the binary expansion of each integer between 1000 and 2000 to determine what $x_i$'s to sum. Hence, $\min n = 11$.
\end{solution*}

\begin{question}\label{A::2024-S-2-4}
    Suppose $p$ is a prime number and $x$, $y$, $z$ are integers satisfying $0 < x < y < z < p$. If $x^3$, $y^3$, $z^3$ have equal remainders when divided by $p$, prove that $x^2 + y^2 + z^2$ is divisible by $x + y + z$.
\end{question}

\begin{question}\label{A::2024-S-2-5}
    Let $a_1, a_2, \ldots$ be a sequence of positive numbers satisfying, for any positive integers $k$, $l$, $m$, $n$ such that $k + n = m + l$, \[\frac{a_k + a_n}{1 + a_k a_n} = \frac{a_m + a_l}{1 + a_m a_l}.\] Show that there exist positive numbers $b$, $c$ so that $b \leq a_n \leq c$ for any positive integer $n$.
\end{question}
\begin{solution*}[\credit{https://www.youtube.com/watch?v=Mrjm7aWXoNQ}{Way Tan}]
    Taking reciprocals, we have \[\frac{1 + a_k a_n}{a_k + a_n} = \frac{1 + a_m a_l}{a_m + a_l} \iff \frac{1 + a_k + a_n + a_k a_n}{a_k + a_n} = \frac{1 + a_m + a_l + a_m a_l}{a_m + a_l}.\] The numerators factor nicely as \[\frac{(1 + a_k)(1 + a_n)}{a_k + a_n} = \frac{(1 + a_m)(1 + a_l)}{a_m + a_l}.\] Taking reciprocals once again, we arrive at the set-up of \hyperref[A::2020-O-1-19]{2020/Open/R1/Q19}: \[\frac{a_k + a_n}{(1 + a_k)(1 + a_n)} = \frac{a_m + a_l}{(1 + a_m)(1 + a_l)}.\] Letting $b_i = \frac{1}{a_i + 1} - \frac12$ as given in the hint to that question, we have after simplification, \[b_kb_n = b_m b_n.\] However, by the condition that $k + n = m + l$, we clearly have that $\bc{b_i}$ is in geometric progression. Moreover, because $a_i \in (0, \infty)$, we have $b_i \in (-\frac12, \frac12)$. Since $b_i$ is bounded, it follows that $\abs{r} \leq 1$, where $r$ is the common ratio of $\bc{b_i}$. It follows that $\abs{b_i} \leq \abs{b_1} < \frac12 $ for all positive integers $i$. Hence, $-\abs{b_1} \leq b_i \leq \abs{b_1}$, implying that \[\frac1{\frac12 + \abs{b_1}} - 1 \leq a_i \leq \frac1{\frac12-\abs{b_1}} - 1,\] whence $b = \frac1{\frac12 + \abs{b_1}} - 1$ and $c = \frac1{\frac12-\abs{b_1}} - 1$ are clearly positive, and we are done. \qed
\end{solution*}