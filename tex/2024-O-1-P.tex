\subsection{Round 1 Problems}

Solutions can be found in Section~\ref{S::2024-O-1}.

\begin{enumerate}
    \hyperrefitem[A::2024-O-1-1] Let $S_k = 1 + 2 + 3 + \cdots + k$ for any positive integer $k$. Find $S_1 + S_2 + S_3 + \cdots + S_{20}$.
    \hyperrefitem[A::2024-O-1-2] Let $S = \sum_{r=1}^{64} r\binom{64}{r}$, where $\binom{n}{r} = \frac{n!}{r! (n-r)!}$ and $0! = 1$. Find $\log_2 S$.
    \hyperrefitem[A::2024-O-1-3] Let $x$ be the largest number in the interval $[0, 2\pi]$ such that $(\sin x)^{2024} - (\cos x)^{2024} = 1$. Find $\floor x$.

    \textit{(Note: If you think that such a number $x$ does not exist, enter your answer ``99999''.)}
    \hyperrefitem[A::2024-O-1-4] Find the number of real numbers $x$ that satisfies the equation $\abs{x - 2} + \abs{x - 3} = \abs{2x - 5}$.

    \textit{(Note: If you think that there are no such numbers, enter ``0''; if you think that there are infinitely many such numbers, enter ``99999''.)}
    \hyperrefitem[A::2024-O-1-5] Among all the real numbers that satisfies the inequality $e^x \geq 1 + 2e^{-x}$, find the minimum value of $\ceil{e^x + e^{-x}}$.
    \hyperrefitem[A::2024-O-1-6] Find the smallest positive integer $C$ greater than 2024 such that the sets $A = \bc{2x^2 + 2x + C : x \in \ZZ}$ and $B = \bc{x^2 + 2024x + 2: x \in \ZZ}$ are disjoint.
    \hyperrefitem[A::2024-O-1-7] Let $ABCD$ be a convex quadrilateral inscribed in a circle $\o$. The bisector of $\angle BAC$ meets $\o$ at $E$ ($\neq A$), the bisector of $\angle ABD$ meets $\o$ at $F$ ($\neq B$), $AE$ intersects $BF$ at $P$ and $CF$ intersects $DE$ at $Q$. Suppose $EF = 20$, $PQ = 11$. Find the area of the quadrilateral $PEQF$.
    \hyperrefitem[A::2024-O-1-8] Let $f(x) = \sqrt{x^2 + 1} + \sqrt{(4-x)^2 + 4}$. Find the minimum value of $f(x)$.
    \hyperrefitem[A::2024-O-1-9] It is known that $a \geq 0$ satisfies $\sqrt{4 + \sqrt{4 + \sqrt{4 + \sqrt{4 + a}}}} = a$. Find the value of $(2a - 1)^2$.
    \hyperrefitem[A::2024-O-1-10] A rectangle with sides parallel to the horizontal and vertical axes is inscribed in the region bounded by the graph of $y = 60 - x^2$ and the $x$-axis. If the area of the largest such rectangle has area $k\sqrt 5$, find the value of $k$.
    \hyperrefitem[A::2024-O-1-11] Let $x$ be a real number satisfying the equation $x^{x^5} = 100$. Find the value of $\floor{x^5}$.
    \hyperrefitem[A::2024-O-1-12] Let $a$, $b$, $c$, $d$, $e$ be distinct integers with $a + b + c + d + e = 9$. If $m$ is an integer such that \[(m-a)(m-b)(m-c)(m-d)(m-e) = 2009,\] determine the value of $m$.
    \hyperrefitem[A::2024-O-1-13] Let $\bc{x}$ be the fractional part of the number $x$, i.e., $\bc{x} = x - \floor{x}$. If $S = \int_0^9 \bc{x}^2 \d x$, find $\floor{S}$.
    \hyperrefitem[A::2024-O-1-14] The solution of the inequality $\abs{(x+1)(x-6)} > \abs{(x+4)(x-2)}$ can be expressed as $x < a$ or $b < x < c$. If $S = \abs{a} + \abs{b} + \abs{c}$, find $\floor{14S}$.
    \hyperrefitem[A::2024-O-1-15]Given that $x, y > 0$ and $x\sqrt{2-y^2} + y\sqrt{2-x^2} = 2$, find the value of $x^2 + y^2$.
    \hyperrefitem[A::2024-O-1-16] A convex polygon has $n$ sides such that no three diagonals are concurrent. It is known that all its diagonals divide the polygon into 2500 regions. Determine $n$.
    \hyperrefitem[A::2024-O-1-17] Find the number of integers $n$ between $-2029$ and 2029 inclusive such that $(n+2)^2 + n^2$ is divisible by 2029.
    \hyperrefitem[A::2024-O-1-18] Let $f$ be a function such that for any real number $x$, we have $f(x) + 2f(2-x) = x + x^2$. Find the value of $f(1) + f(2) + f(3) + \cdots + f(34)$.
    \hyperrefitem[A::2024-O-1-19] Find the largest possible positive prime integer $p$ such that $p$ divides \[S(p) = 1^{p-2} + 2^{p-2} + 3^{p-2} + 4^{p-2} + 5^{p-2} + 6^{p-2} + 7^{p-2} + 8^{p-2}.\]
    \hyperrefitem[A::2024-O-1-20] Let $f$ be a function such that $f(x) + f(\frac{1}{1-x}) = 1 + \frac1x$ for all $x \notin \bc{0, 1}$. Find the value of $\floor{180 \cdot f(10)}$.
    \hyperrefitem[A::2024-O-1-21] Let $C$ be a circle with equation $(x-a)^2 + (y-b)^2 = r^2$, where at least one of the $a$ and $b$ are irrational numbers. Find the maximum possible numbers of points $(p,q)$ on $C$ where both $p$ and $q$ are rational numbers.
    \hyperrefitem[A::2024-O-1-22] On the plane there are 2024 points coloured either red or blue such that each red point is the centre of a circle passing through 3 blue points. Determine the least number of blue points.
    \hyperrefitem[A::2024-O-1-23] It is given that the positive real numbers $x_1, \ldots, x_{2026}$ satisfy $\dfrac{x_1^2}{x_1^2 + 1} + \cdots + \dfrac{x_{2026}^2}{x_{2026}^2 + 1} = 2025$. Find the maximum value of $\dfrac{x_1}{x_1^2 + 1} + \cdots + \dfrac{x_{2026}}{x_{2026}^2 + 1}$.
    \hyperrefitem[A::2024-O-1-24] Let $n$ denote the number of ways of arranging all the letters of the word MATHEMATICS in one row such that
    \begin{itemize}
        \item both M's precede both T's; and
        \item neither the two M's nor the two T's are next to each other.
    \end{itemize}
    Determine the value of $\frac{n}{6!}$.
    \hyperrefitem[A::2024-O-1-25] The incircle of the triangle $ABC$ centred at $I$ touches the sides $BC$, $CA$, $AB$ at $D$, $E$, $F$ respectively. Let $D'$ be the intersection of the extension of $ID$ with the circle through $B$, $I$, $C$; $E'$ be the intersection of the extension of $IE$ with the circle through $A$, $I$, $C$; and $F'$ the intersection of the extension of $IF$ with the circle through $A$, $I$, $B$. Suppose $AB = 52$, $BC = 56$, $CA = 60$. Find $DD' + EE' + FF'$.
 \end{enumerate}
