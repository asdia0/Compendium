\subsection{Round 1 Solutions}\label{S::2023-O-1}

\resph{https://www.youtube.com/watch?v=mWaub25ruXE}{Review by \textit{Way Tan}}

\begin{question}[10]\label{A::2023-O-1-1}
    The graph $C$ with equation $y = \dfrac{ax^2 + bx + c}{x+2}$ has an oblique asymptote with equation $y = 4x - 6$ and one of the stationary points at $x = -4$. Find the value of $a + b + c$.
\end{question}

Since $C$ has an oblique asymptote with equation $y = 4x - 6$, its equation can be written as \[C: y = 4x - 6 + \frac{d}{x+2}\] for some $d \in \RR$. Multiplying throughout by $x+2$ and comparing coefficients, we get $a = 4$, $b = 2$ and $c = -12 + d$. Differentiating and using the fact that $\derx{y}{x} = 0$ at $x = -4$ yields $d = 16$. Thus, $a + b + c = 10$.

\begin{question}[2000]\label{A::2023-O-1-2}
    If $x = \dfrac{1}{1} + \dfrac{1}{1+2} + \dfrac{1}{1+2+3} + \dfrac{1}{1+2+3+4} + \cdots + \dfrac{1}{1 + 2 + 3 + \cdots + 100}$, find the value of $\floor{1010x}$.
\end{question}

Observe that the denominators are the triangular numbers. It is well known that the $n$th triangular number is given by $n(n+1)/2$. Thus, each term of $x$ is of the form $2/[n(n+1)]$, which can be written as $2/n - 2/(n+1)$ via partial fraction decomposition. $x$ is hence a telescoping sum, which evaluates to $2 - 2/101$, whence $1010x = 2020 - 20 = 2000$.

\begin{question}[7]\label{A::2023-O-1-3}
    The set of all possible values of $x$ for which the sum of the infinite series \[1 + \dfrac16 \bp{x^2 - 5x} + \dfrac1{6^2} \bp{x^2 - 5x}^2 + \dfrac1{6^3} \bp{x^2 - 5x}^3 + \cdots\] exists can be expressed as $(a, b) \cup (c, d)$, where $a < b < c < d$. Find $d - a$.
\end{question}

Observe that the common ratio of the given infinite series is $r = (x^2 - 5x)/6$. For the sum to exist, $\abs{r} < 1$, whence $-6 < x^2 - 5x < 6$. Since we are interested in the extreme values of $x$, we consider only the equality case. We thus obtain $x^2 - 5x = -6$ or $x^2 - 5x = 6$, which clearly has solutions $x = -1, 2, 3, 6$, giving $d - a = 6 - (-1) = 7$.

\clearpage
\begin{question}[2]\label{A::2023-O-1-4}
    Find the value of $\floor{y}$, where $y = \displaystyle\sum_{k=0}^\infty (2k+1)(0.5)^{2k}$.
    
    \noindent\textit{(Hint: Consider the series expansion of $(1 - x)^{-2}$)}
\end{question}

Note that \[\frac1{1-x} = \sum_{k=0}^\infty x^k.\] Differentiating yields \[\dfrac1{(1-x)^2} = \sum_{k=0}^\infty kx^{k-1} = x^{-1} \sum_{k=0}^\infty k x^k.\] Now observe $y$ can be rewritten in terms of the above series: \[y = 2\sum_{k=0}^\infty k (1/4)^{k} + \sum_{k=0}^\infty (1/4)^k,\] from which it clearly follows that \[\floor y = \floor{\frac{2 \cdot 1/4}{(1 - 1/4)^2} + \frac1{1 - 1/4}} = 2.\]
    
\begin{question}[3]\label{A::2023-O-1-5}
    The solution of the inequality $\abs{x-1} + \abs{x+1} < ax + b$ is $-1 < x < 2$. Find the value of $\floor{a + b}$.
\end{question}

At the extreme ends of the solution interval, equality is achieved. This yields $2 = -a + b$ and $4 = 2a + b$ upon substituting $x = -1$ and $x = 2$ into the two expressions. Solving, we get $a = 2/3$ and $b = 8/3$, whence $\floor{a + b} = 3$.
    
\begin{question}[32]\label{A::2023-O-1-6}
    The equation $x^4 - 4x^2 + qx - r = 0$ has three equal roots. Find the value of $\floor{\dfrac{3q^2}{r^2}}$.
\end{question}

Let $\a$ be the root of multiplicity 3 and $\b$ be the remaining root. By Vieta's formulas, we have the following system of equations:
\[\left\{\begin{aligned}
    -r &= \a^3 \b\\
    -q &= \a^3 + 3\a^2 \b\\
    -4 &= 3\a^2 + 3\a\b\\
    0 &= 3\a + \b
\end{aligned}\right.\]
From the third and fourth equations, we have $\a \b = -2$. From the second and third equations, we have $-4\a + q = 2\a^3$. However, $q = -\a^3 - 3\a (\a \b) = -\a^3 + 6\a$. Combining equations gives $3\a^3 - 2\a = 0$, whence $\a = \sqrt{2/3}$, since $\a$ is clearly non-zero. Thus,
\begin{align*}
    \floor{\frac{3q^2}{r^2}} &= \floor{\frac{3(-\a^3 + 6\a)^2}{(-2\a^2)^2}} = \floor{\frac{3(\a^6 - 12 \a^4 + 36\a^2)}{4\a^4}}\\ &= \floor{\frac{3((2/3)^3 - 12(2/3)^2 + 36(2/3))}{4(2/3)^2}} = 32
\end{align*}

\begin{question}[8]\label{A::2023-O-1-7}
    The parabolas $y = x^2 - 16x + 50$ and $x = y^2$ intersect at 4 distinct points which lie on a circle centred at $(a, b)$. Find $\abs{a - b}$.
\end{question}

From the first equation, we have $x^2 - 17x + x - y = -50$. By the second equation, this is equivalent to $x^2 - 17x + y^2 - y = -50$. Completing the square, we obtain \[\bp{x - \frac{17}2}^2 + \bp{y - \frac12}^2 = -50 + \bp{\frac{17}{2}}^2 + \bp{\frac12}^2.\] It is thus clear that $a = 17/2$ and $b = 1/2$, giving $\abs{a - b} = 8$.

\begin{question}[33]\label{A::2023-O-1-8}
    In the 3-dimensional Euclidean space with origin $O$ and three mutually perpendicular $x$-, $y$- and $z$-axes, two planes $x + y + 3z = 4$ and $2x - z = 6$ intersect at the line $\vec r \crossp \cveciii{-1}{a}{b} = \cveciii{-2}{c}{d}$. Find the value of $\abs{a + b + c + d}$.
\end{question}

Solving the Cartesian equations of the two planes simultaneously, we get \[\vec r = \dfrac17 \cveciii{22-\l}{7\l}{2-2\l},\] where $\l \in \RR$. Taking the cross product yields \[\dfrac17 \cveciii{-2a + \l(2a+7b)}{-2-22b + \l(2+b)}{22a+\l(7-a)} = \cveciii{-2}{c}{d}\] Since the above equation must hold for all real $\l$, we immediately get $a = 7$ and $b = -2$. It quickly follows by equating the $\hat {\vec j}$ and $\hat{\vec k}$ components of both vectors that $c = 6$ and $d = 22$, giving $\abs{a + b + c + d} = 33$.

\begin{question}[200]\label{A::2023-O-1-9}
    Let $x$, $y$, $z$ be real numbers with $3x + 4y + 5z = 100$. Find the minimum value of $x^2 + y^2 + z^2$.
\end{question}

Observe that $3x + 4y + 5z = 100$ describes a plane $\pi$ in 3-dimensional Euclidean space with vector equation \[\pi: \vec r \cdot \cveciii345 = 100.\] Observe also that $\min(x^2 + y^2 + z^2)$ is the square of the perpendicular distance between the origin and $\pi$. Applying the standard formula for perpendicular distance between a plane and a point, one gets \[\min(x^2 + y^2 + z^2) = \bp{\dfrac{100}{\sqrt{3^2 + 4^2 + 5^2}}}^2 = 200.\]

\begin{question}[2]\label{A::2023-O-1-10}
    Find the area of the region represented by the equation $\floor{x} + \floor{y} = 1$ in the region $0 \leq x < 2$.

    \noindent\textit{(Note: If you think that there is no area defined by the graph, enter ``0''; if you think that the area is infinite, enter ``9999''.)}
\end{question}

When $x \in [0, 1)$, we have $\floor x = 0$. Thus, $\floor y = 1$, giving $y \in [1, 2)$. This is a square of area 1. Similarly, when $x \in [1, 2)$, we have $\floor x = 1$. Thus, $\floor y = 0$, giving $y \in [0, 1)$. This is another square of area 1. Hence, the total area of the region is 2.

\begin{question}[1011]\label{A::2023-O-1-11}
    Let $ABC$ be a triangle satisfying the following conditions that $\angle A + \angle C = 2\angle B$, and $\dfrac1{\cos A} + \dfrac1{\cos C} = \dfrac{-\sqrt2}{\cos B}$. Determine the value of $\dfrac{2022\cos{\frac{A-C}{2}}}{\sqrt2}$.
\end{question}

\sol{1} Note that $\angle A + \angle B + \angle C = 180 \deg$, whence $\angle B = 60 \deg$. Clearing denominators in the given equation, we have \[\cos A + \cos B = -2\sqrt{2}\cos A \cos C. \] Without loss of generality, let $\angle A = 60\deg + \t$ and $\angle C = 60 \deg - \t$. We now aim to find $\cos{\frac{A-C}{2}} = \cos\t$. We have \[\cos{60\deg + \t} + \cos{60\deg - \t} = -2\sqrt2 \cos{60\deg + \t}\cos{60 \deg - \t}.\] Expanding using cosine identities yields \[4\sqrt2 \cos^2 \t + 2\cos \t - 3\sqrt2 = 0,\] which has the unique solution $\cos \t = 1/\sqrt2$ (keeping in mind $\abs{\cos \t} \leq 1$). The desired expression thus evaluates to \[\frac{2022\cos{\tfrac{A-C}{2}}}{\sqrt2} = \frac{2022/\sqrt{2}}{\sqrt2} = 1011.\]

\sol{2}[Abusing integers] In order for $\frac{2022\cos{\frac{A-C}{2}}}{\sqrt2}$ to be an integer, we need $\cos{\frac{A-C}{2}}$ to be of the form $k\sqrt{2}$, where $k$ is a positive rational such that $2022k$ is an integer. It is exceedingly likely that $\frac{2022\cos{\frac{A-C}{2}}}{\sqrt2} = \frac{\sqrt2}{2}$, as it is a special value of the cosine function. The required answer is thus $\frac{2022\cdot\sqrt{2}/2}{\sqrt2} = 1011$.

\begin{question}[2020]\label{A::2023-O-1-12}
    Find $x$ which satisfies the following equation \[\dfrac{x-2019}{1} + \dfrac{x-2018}{2} + \dfrac{x-2017}{3} + \cdots + \dfrac{x+2}{2022} + \dfrac{x+3}{2023} = 2023.\]
\end{question}

By inspection, 2020 is clearly a solution, as there are 2023 terms and each term evaluates to 1, giving a sum of 2023 as desired.

\begin{question}[229]\label{A::2023-O-1-13}
    Assume that $x$ is a positive number such that $x - \frac1x = 3$ and \[\dfrac{x^{10} + x^8 + x^2 + 1}{x^{10} + x^6 + x^4 + 1} = \dfrac{m}{n},\] where $m$ and $n$ are positive integers without common factors larger than 1. Determine the value of $m + n$.
\end{question}

Observe that \[\frac{x^{10} + x^8 + x^2 + 1}{x^{10} + x^6 + x^4 + 1} = \frac{x^8 + 1}{x^4 + 1} \cdot \frac{x^2+1}{x^6+1} = \frac{x^4\bp{x^4 + x^{-4}}}{x^2\bp{x^2 + x^{-2}}} \cdot \frac{x\bp{x+x^{-1}}}{x^3\bp{x^3+x^{-3}}} = \frac{x^4 + x^{-4}}{x^2 + x^{-2}} \cdot \frac{x+x^{-1}}{x^3+x^{-3}}\]
Repeatedly squaring $x - x^{-1} = 3$ yields
\begin{align*}
    x^2 + x^{-2} &= 3^2 + 2 = 11,\\
    x^4 + x^{-4} &= 11^2 - 2 = 119.
\end{align*}
Now observe that \[x^3 + x^{-3} = \bp{x + x^{-1}}\bp{x^2 + x^{-2} - 1} = 10\bp{x + x^{-1}}.\] The given expression thus evaluates to \[\frac{x^4 + x^{-4}}{x^2 + x^{-2}} \cdot \frac{x+x^{-1}}{x^3+x^{-3}} = \frac{119}{11} \cdot \frac1{10} = \frac{119}{110},\] whence $m + n = 119 + 110 = 229$.

\begin{question}[24]\label{A::2023-O-1-14}
    Consider the set of all possible pairs $(x, y)$ of real numbers that satisfy $(x-4)^2 + (y-3)^2 = 9$. If $S$ is the largest possible value of $\dfrac{y}{x}$, find the value of $\floor{7S}$.
\end{question}

Observe that $(x-4)^2 + (y-3)^2 = 9$ describes a circle with centre $(4, 3)$ and radius 3. Also observe that $y/x$ is the gradient of the line passing through the origin and some point on the circle with coordinates $(x, y)$. The largest possible value of $y/x$ hence occurs when the line in question is tangent to the circle. Consider the simultaneous equations $(x-4)^2 + (y-3)^2 = 9$ and $S = y/x$. This combines to give $(x-4)^2 + (Sx - 3)^2 = 9$. Expanding, we have $(1+S^2)x^2 - (8+6S)x + 16 = 0$. Since the line is tangent to the circle, there is only one solution. The discriminant of the above quadratic is hence 0, giving $(8+6S)^2 - 4(1+S^2)(16) = 0$. Solving, we get $S = 24/7$, whence $\floor {7S} = 24$.

\begin{question}[46]\label{A::2023-O-1-15}
    Let $x$, $y$ be positive integers with $16x^2 + y^2 + 7xy \leq 2023$. Find the maximum value of $4x + y$.
\end{question}

Let $k$ be the maximum value of $4x + y$ without the restriction of $x$ and $y$ being integers. Then the line $4x + y = k$ is tangent to the elliptical region given by $16x^2 + y^2 + 7xy \leq 2023$. Equating the two gives \[16x^2 + (k-4x)^2 + 7x(k-4x) = 2023 \implies 4x^2 - kx + (k^2 - 2023) = 0.\] Setting the discriminant to 0, we get $k^2 - 16(k^2 - 2023) = 0$, whence $k^2 = 2023 \cdot 16 / 15$. Reinstating the integral restriction on $x$ and $y$, we get $k = 46$, which can indeed be achieved (e.g. $x = 5$, $y = 26$).

\begin{question}[25]\label{A::2023-O-1-16}
    Let $x$ be the largest real number such that \[\sqrt{x - \dfrac1x} + \sqrt{1 - \dfrac1x} = x.\] Determine the value of $(2x-1)^4$.
\end{question}

Clearing denominators, we have \[\sqrt{x^2 - 1} + \sqrt{x - 1} = x^{3/2} \implies \sqrt{x^2 - 1} = x^{3/2} - \sqrt{x-1}.\] Squaring both sides yields \[x^2 - 1 = x^3 + x - 1 - 2x^{3/2}\sqrt{x-1} \implies 2x^{1/2}\sqrt{x-1} = x^2 - x + 1.\] Squaring once again yields \[4x(x-1) = x^4 - 2x^3 + 3x^2 - 2x + 1 \implies x^4 - 2x^3 - x^2 + 2x + 1 = 0.\] Now note that \begin{align*}
    (2x-1)^4 &= 16x^4 - 32 x^3 + 24x^2 - 8x + 1\\
    &= 16(x^4 - 2x^3 -x^2 + 2x + 1) + 10(4x^2 - 4x + 1) + 25\\
    &= 10(2x-1)^2 - 25
\end{align*}
whence $(2x-1)^2 = 5$ and thus $(2x-1)^4 = 25$.
   
\begin{question}[64]\label{A::2023-O-1-17}
    Two positive integers $m$ and $n$ differ by 10 and the digits in the decimal representation of $mn$ are all equal to 9. Determine $m + n$.
\end{question}

By inspection, $999 = 27 \cdot 37$. Thus, $m + n = 27 + 37 = 64$.
    
\begin{question}[1]\label{A::2023-O-1-18}
    Let $\bc{a_n}$ be a sequence of positive numbers, and let $S_n = a_1 + a_2 + a_3 + \cdots + a_n$. For any positive integer $n$, let $b_n = \dfrac12 \bp{\dfrac{a_{n+1}}{a_n} + \dfrac{a_n}{a_{n+1}}}$. Given that $\dfrac{a_n + 2}{2} = \sqrt{2S_n}$ holds for all positive integers $n$, determine the limit $\lim_{n \to \infty} (b_1 + b_2 + \cdots + b_n - n)$.
\end{question}

We claim that $a_n = 4n - 2$. When $n = 1$, $S_1 = a_1$, whence it is clear that \[\frac{a_1 + 2}{2} = \sqrt{2a_1} \implies a_1 = 2,\] satisfying our claim. Now assume that $a_k = 4k - 2$ for some $k \in \NN$. We have \[S_k = \sum_{n = 1}^k (4n - 2) = 2k^2.\] From the given condition, \[\frac{a_{k+1} + 2}{2} = \sqrt{2 \bp{2k^2 + a_{k+1}}} \implies a^2_{k+1} - 4a_{k+1} + 4 - 16k^2 = 0,\] which has the unique positive solution $a_{k+1} = 4k + 2 = 4(k+1) - 2$. This closes the induction.

$b_n$ thus simplifies to \[b_n = \frac12 \bp{\dfrac{a_{n+1}}{a_n} + \dfrac{a_n}{a_{n+1}}} = \frac12 \cdot \frac{a^2_{n+1} + a^2_n}{a_n a_{n+1}} = \frac{4n^2 + 1}{4n^2 - 1} = 1 + \frac1{2n-1} - \frac{1}{2n+1}.\] The limit in question hence telescopes to $1$.
    
\begin{question}[20]\label{A::2023-O-1-19}
    Let $ABC$ be a triangle with $AB = c$, $AC = b$ and $BC = a$, and satisfies the conditions $\tan C = \dfrac{\sin A + \sin B}{\cos A + \cos B}$, $\sin{B - A} = \cos C$ and that the area of triangle $ABC = 3 + \sqrt{3}$. Determine the value of $a^2 + c^2$.
\end{question}

By the sum-to-product formulae, we have \[\tan C = \frac{2\sin{\frac{A+B}{2}} \cos{\frac{A-B}{2}}}{2\cos{\frac{A+B}{2}} \cos{\frac{A-B}{2}}} = \tan{\tfrac{A+B}{2}}.\] Hence, $C = (A+B)/2$, implying that $\angle C = 60\deg$. Let $\angle A = 60\deg - \t$ and $\angle B = 60\deg + \t$. Then $\sin 2\t = \cos C = 1/2$, whence $\t = 15\deg$, thus $\angle A = 45\deg$ and $\angle C = 75\deg$. We now express the area of the triangle in terms of the side lengths: \[3 + \sqrt{3} = \frac12 ab \sin C = \frac12 bc \sin A = \frac12 ac \sin B.\] It quickly follows that \[\frac{a}{c} = \sqrt{\frac23}, \quad ac = \frac{2(3 + \sqrt3)}{\sin B}.\] This immediately gives us $a^2$ and $c^2$ in terms of $\sin B$: \[a^2 = \sqrt{\frac23} \cdot \frac{2(3 + \sqrt3)}{\sin B}, \quad c^2 = \sqrt{\frac32} \cdot \frac{2(3 + \sqrt3)}{\sin B}.\] Since $\sin B = \sin{30\deg + 45\deg} = \sqrt{2}(1 + \sqrt3)/4$, we finally get \[a^2 + c^2 = \frac{2(3 + \sqrt3)}{\sqrt{2}(1 + \sqrt3)/4} \bp{\sqrt{\frac23} + \sqrt{\frac32}} = 20.\]
    
\begin{question}\label{A::2023-O-1-20}
    \textbf{[VOID]} Let $g \colon \RR \to \RR$, $g(0) = 4$ and that \[g(xy + 1) = g(x)g(y) - g(y) - x + 2023.\] Find the value of $g(2023)$.
\end{question}

We will show that there are two different expressions for $g(x)$ that result in different answers to $g(2023)$.

Firstly, let $x = 0$. Then \[g(1) = g(0)g(y) - g(y) - 0 + 2023 \implies g(y) = \frac{g(1) - 2023}{3},\] which is constant. Letting $y = 1$, we get $g(1) = -2023/2$, whence $g(2023) = -2023/2$.

Secondly, let $y = 0$. Then \[g(1) = g(x)g(0) - g(0) - x + 2023 \implies g(x) = \frac{g(1) + x - 2019}{4}.\] Letting $x = 1$, we get $g(1) = -2018/3$, whence $g(2023) = -1003/6$, a contradiction.

Hence, $g$ does not exist, and the question is void.

\begin{question}[7]\label{A::2023-O-1-21}
    In the triangle $ABC$, $D$ is the midpoint of $AC$, $E$ is the midpoint of $BD$, and the lines $BA$ and $CE$ are tangent to the circumcircle of the triangle $ADE$ at $A$ and $E$ respectively. Suppose the circumradius of the triangle $AED$ is $(\frac{64}{7})^{1/4}$. Find the area of the triangle $ABC$.
\end{question}

\begin{center}
    \begin{tikzpicture}[rotate=-90, scale=1.3]
        \coordinate[label=left:$A$] (A) at (0, 0);
        \coordinate[label=left:$B$] (B) at (-2.300, 0);
        \coordinate[label=right:$C$] (C) at (-2.300, 6.086);
        \coordinate[label=below:$D$] (D) at (-1.15, 3.043);
        \coordinate[label=below:$E$] (E) at (-1.725, 1.5215);

        \draw (A) -- (B);
        \draw (B) -- (C);
        \draw (C) -- (A);
        \draw (B) -- (D);
        \draw (E) -- (C);
        \draw (A) -- (E);

        \draw[dashed] (0, 1.7389) circle[radius=1.7389];

        \draw pic [draw, angle radius=5mm, ""] {angle = D--E--C};

        \draw pic [draw, angle radius=5mm, ""] {angle = D--A--E};

        \draw pic [draw, angle radius=5mm, ""] {angle = E--D--A};
        \draw pic [draw, angle radius=4mm, ""] {angle = E--D--A};

        \draw pic [draw, angle radius=4.5mm, ""] {angle = E--A--B};
        \draw pic [draw, angle radius=5.5mm, ""] {angle = E--A--B};

        \node[anchor=north] at ($(A)!0.5!(D)$) {$x$};
        \node[anchor=north] at ($(D)!0.5!(C)$) {$x$};
        \node[anchor=north] at ($(E)!0.5!(B)$) {$y$};
        \node[anchor=north] at ($(E)!0.5!(D)$) {$y$};

    \end{tikzpicture}
\end{center}

Let $AD = DC = x$ and $BE = ED = y$.

By the power of a point theorem, we have $BA^2 = BE \cdot BD$ and $CE^2 = CD \cdot CA$, whence \[BA = \sqrt{2}y, \quad CE = \sqrt{2}x.\] By the alternate segment theorem, we have $\angle CED = \angle EAD$ and $\angle BAE = \angle BDA$. Hence, $\triangle BAE$ is similar to $\triangle BDA$, and $\triangle CED$ is similar to $\triangle CAE$. Thus, \[\frac{BE}{BA} = \frac{AE}{DE} \implies AE = \frac{x}{\sqrt2}, \quad \frac{CD}{CE} = \frac{ED}{AE} \implies AE = \sqrt{2}y.\] It follows that $x = 2y$. Using the cosine rule on $\triangle ADE$, we obtain $\cos \angle ADE = 3/4$, whence $\sin \angle ADE = \sqrt{7}/4$. By the extended sine rule, \[\frac{\sqrt{2}y}{\sqrt{7}/{4}} = 2\bp{\frac{64}{7}}^{1/4} \implies y = 7^{1/4}.\] We finally get \[[ABC] = 2[ABD] = 4[AED] = 4 \bp{\frac12 \cdot 2y \cdot y \cdot \sin \angle ADE} = 7.\]

\clearpage
\begin{question}[48]\label{A::2023-O-1-22}
    $ABCD$ is a parallelogram such that $\angle ABC < 90\deg$ and $\sin \angle ABC = \frac45$. The point $K$ is on the extension of $BC$ such that $DC = DK$; the point $L$ is on the extension of $DC$ such that $BC = BL$. The bisector of $\angle CDK$ intersects the bisector of $\angle LBC$ at $Q$. Suppose the circumradius of the triangle $ABD$ is 25. Find the length of $KL$.
\end{question}

\begin{center}
    \begin{tikzpicture}[scale=0.4]
        \coordinate[label=above left:$A$] (A) at (3, 4);
        \coordinate[label=below left:$B$] (B) at (0, 0);
        \coordinate[label=below right:$C$] (C) at (8, 0);
        \coordinate[label=above:$D$] (D) at (11, 4);
        \coordinate[label=below:$K$] (K) at (14, 0);
        \coordinate[label=below right:$L$] (L) at (2.24, -7.68);
        \coordinate[label=below:$M$] (M) at (11, 0);
        \coordinate[label=below right:$N$] (N) at (5.12, -3.84);

        \draw (A) -- (B);
        \draw (B) -- (K);
        \draw (L) -- (D);
        \draw (D) -- (A);
        \draw (D) -- (K);
        \draw (B) -- (L);
        \draw[dashed] (B) -- (N);
        \draw[dashed] (D) -- (M);

        \draw pic [draw, angle radius=5mm, ""] {angle = C--B--A};
        \draw pic [draw, angle radius=4mm, ""] {angle = C--B--A};

        \draw pic [draw, angle radius=5mm, ""] {angle = K--C--D};
        \draw pic [draw, angle radius=4mm, ""] {angle = K--C--D};

        \draw pic [draw, angle radius=5mm, ""] {angle = B--C--N};
        \draw pic [draw, angle radius=4mm, ""] {angle = B--C--N};

        \draw pic [draw, angle radius=2mm, ""] {right angle = C--M--D};
    
        \draw pic [draw, angle radius=2mm, ""] {right angle = C--N--B};

        \node[anchor=south] at ($(A)!0.5!(D)$) {$x$};
        \node[anchor=south east] at ($(A)!0.5!(B)$) {$y$};

    \end{tikzpicture}
\end{center}

Let $x$ and $y$ be the lengths $AD$ and $BC$ respectively. Let $M$ be the intersection between the bisector of $\angle CDK$ and $BK$; let $N$ be the intersection between the bisector of $\angle LBC$ and $DL$.

Observe that \[[\triangle ABD] = \frac{xy \cdot BD}{4 \cdot 25} = \frac12 xy \sin \angle ABC \implies BD = 40.\] Note that $\cos \angle ABC = 3/5$ and $\cos \angle BCD = -3/5$ (since $\angle BCD = 180\deg - \angle ABC$). Using the cosine rule on $\triangle BCD$, we obtain \[x^2 + y^2 + \frac65 xy = 40^2.\] Since $\angle BNC = \angle DMC = 90 \deg$, and $\angle BCN = \angle DCM = \angle ABC$, by AAA, we have that $\triangle BNC$ is similar to $\triangle DMC$, $\triangle BNC \equiv \triangle BNL$, $\triangle BMC \equiv \triangle BMK$, and that \[BN = \frac45 x, \quad NC = \frac35 x, \quad DM = \frac45 y, \quad MC = \frac35 y.\] It also follows that $BDMN$ is cyclic. Applying Ptolemy's theorem, we have \[KL \cdot 40 + xy = \bp{x + \frac65 y}\bp{y + \frac65 x}.\] Expanding, we finally have \[KL = \frac1{40} \cdot \frac65 \bp{x^2 + y^2 + \frac65 xy} = 48.\]

\clearpage
\begin{question}[7]\label{A::2023-O-1-23}
    A group of 200 monkeys is given the task of picking up all 3000 peanuts on the ground. Determine the maximum number $k$ such that there must be $k$ monkeys picking up the same number of peanuts. (It is possible that some lazy monkeys may not pick up any peanuts at all).
\end{question}

Consider the worst-case scenario, where there are $n$ monkeys picking 0 peanuts, $n$ monkeys picking 1 peanut, etc. The total number of peanuts picked can be calculated as \[\ \underbrace{n\bp{0 + 1 + 2 + \cdots + \bp{\floor{\frac{200}{n}} - 1}}}_{\text{$n\floor{200/n}$ monkeys}} + \underbrace{\floor{\frac{200}{n}} + \bp{\floor{\frac{200}{k}} + 1} + \cdots}_{\text{remaining monkeys}}.\] The smallest value of $n$ where the above expression is less than 3000 is $7$. Hence, $k = 7$.

\begin{question}[7]\label{A::2023-O-1-24}
    A chain of $n$ identical circles $C_1, C_2, \ldots, C_n$ of equal radii and centres on the $x$-axis lie inside the ellipse $E: \frac{x^2}{2023} + \frac{y^2}{333} = 1$ such that $C_1$ is tangent to $E$ internally at $(-\sqrt{2023}, 0)$, $C_n$ is tangent to $E$ internally at $(\sqrt{2023}, 0)$, and $C_i$ is tangent to $C_{i+1}$ externally for $i = 1, \ldots , n-1$. Determine the smallest possible value of $n$.
\end{question}

The curvature $\k$ of an ellipse $x^2/a^2 + y^2/b^2 = 1$ is given by \[\k(\t) = \frac{ab}{\bp{a^2\sin^2\t + b^2\cos^2\t}^{3/2}}.\] Taking $a^2 = 2023$, $b^2 = 333$ and $\t = 0$, we have that the curvature of $E$ at $(\sqrt{2023}, 0)$ is $\sqrt{2023}/333$. The maximum radius of $C_n$ is thus $333/\sqrt{2023}$ (since $\k = 1/R$). It immediately follows that \[\min n = \ceil{\frac{2\sqrt{2023}}{2 \cdot 333/\sqrt{2023}}} = 7.\]
    
\begin{question}[6]\label{A::2023-O-1-25}
    Let $p > 2023$ be a prime. Determine the number of positive integers $n$ such that $(n-p)^2 + 2023(2023 - 2n - 2p)$ is a perfect square.
\end{question}

Observe that the given expression is nearly a perfect square. Indeed, it can be rewritten as $k^2 = (n-p-2023)^2 - 4 \cdot 2023 p$. We hence obtain \[2^2 \cdot 7 \cdot 17^2 \cdot p = (n-p-2023)^2 - k^2 = (n-p-2023-k)(n-p-2023+k).\] Observe that the two terms on the right have the same parity, thus both must have a factor of 2. Furthermore, because $n-p-2023 + k > n-p-2023-k$, it must be that $n-p-2023+k$ has the factor of $p$ (since $p > 2023$). Since there are 3 remaining factors, there are hence $\perm{3}{2} = 6$ possible $n$.