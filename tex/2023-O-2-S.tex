\clearpage
\subsection{Round 2 Solutions}\label{S::2023-O-2}

\resph{https://simoxmenblog.blogspot.com/2023/09/smo-open-2023-speedrun.html}{Review by \textit{Glen}}

\begin{question}\label{A::2023-O-2-1}
    In a scalene triangle $ABC$ with centroid $G$ and circumcircle $\o$ centred at $O$, the extension of $AG$ meets $\o$ at $M$; lines $AB$ and $CM$ intersect at $P$; and lines $AC$ and $BM$ intersect at $Q$. Suppose the circumcentre $S$ of the triangle $APQ$ lies on $\o$ and $A$, $O$, $S$ are collinear. Prove that $\angle AGO = 90\deg$.
\end{question}

\resqh{https://artofproblemsolving.com/community/c6h3102141p28048175}{AoPS thread}

\begin{center}
    \begin{tikzpicture}[scale=0.8]
        \coordinate[label=above:$A$] (A) at (0, 2);
        \coordinate[label=left:$B$] (B) at (-2, 0);
        \coordinate[label=below right:$C$] (C) at (0.8091, -1.799);
        \coordinate[label=below left:$P$] (P) at (-4, -2);
        \coordinate[label=below right:$Q$] (Q) at (1.782, -5.581);
        \coordinate[label=below left:$M$] (M) at (-0.739, -1.858);
        \coordinate[label=left:$G$] (G) at (-0.370, 0.070);
        \coordinate[label=below:$O$] (O) at (0,0);
        \coordinate[label=below:$S$] (S) at (0, -2);

        \draw[dashed] (O) circle[radius=2];
        \draw[dashed] (S) circle[radius=4];

        \node[anchor=south west] at (1.6, -1.6) {$\o$};
        
        \draw (A) -- (P);
        \draw (A) -- (Q);
        \draw (B) -- (C);
        \draw (P) -- (Q);
        \draw (C) -- (P);
        \draw (B) -- (Q);

        \fill (G) circle[radius=2pt];
        \fill (O) circle[radius=2pt];
        \fill (S) circle[radius=2pt];
        \fill (M) circle[radius=2pt];

        \draw[dotted] (A) -- (S);
        \draw (A) -- (M);    
        \draw (G) -- (O);
        \draw (M) -- (S);

        \tkzMarkSegment[pos=.5,mark=|](A,B);
        \tkzMarkSegment[pos=.5,mark=|](B,P);

        \tkzMarkSegment[pos=.5,mark=||](A,C);
        \tkzMarkSegment[pos=.5,mark=||](C,Q);

        \draw pic [draw, angle radius=1.5mm, ""] {right angle = A--G--O};
        \draw pic [draw, angle radius=1.5mm, ""] {right angle = A--M--S};
    \end{tikzpicture}
\end{center}

Consider the homothety $H$ centred at $A$ that sends $O$ to $S$ (i.e. $H$ has a scale factor of $\frac{AS}{AO} = 2$). Since homotheties preserve circumcircles and circumcentres, it follows that $H$ sends $\triangle ABC$ to $\triangle APQ$. This means that $B$ and $C$ are the midpoints of $AP$ and $AQ$ respectively, whence $M$ is the centroid of $\triangle APQ$. Because $AS$ is a diameter of $\o$, we have $\angle AMS = 90\deg$. However, because homotheties preserve centroids and angles, we immediately get $\angle AGO = \angle AMS = 90\deg$ as desired.

\begin{question}[Yes]\label{A::2023-O-2-2}
    A grid of cells is tiled with dominoes such that every cell is covered by exactly one domino. A subset $S$ of dominoes is chosen. Is it true that at least one of the following two statements is false?
    \begin{enumerate}
        \item There are 2022 more horizontal dominoes than vertical dominoes in $S$.
        \item The cells covered by the dominoes in $S$ can be tiled completely and exactly by L-shaped tetrominoes.
    \end{enumerate}
\end{question}

\resqh{https://artofproblemsolving.com/community/c6h3102143p28048184}{AoPS thread}

\solctt{https://artofproblemsolving.com/community/c6h3102143p28243581}{bxiao31415} Let $(i, j)$ be coloured in the following manner:
\begin{itemize}
    \item 0 if both $i$ and $j$ are even;
    \item 1 if $i$ is odd and $j$ is even;
    \item 2 if $i$ is even and $j$ is odd; and
    \item 3 if both $i$ and $j$ are odd.
\end{itemize}

As an example, the following $4 \times 4$ grid shows the colouring scheme:
\begin{table}[H]
    \centering
    \begin{tabular}{|l|l|l|l|}
    \hline
    0 & 1 & 0 & 1 \\ \hline
    3 & 2 & 3 & 2 \\ \hline
    0 & 1 & 0 & 1 \\ \hline
    3 & 2 & 3 & 2 \\ \hline
    \end{tabular}
\end{table}

Seeking a contradiction, suppose both (1) and (2) are true. Observe that each horizontal domino covers squares whose sum is 1 mod 4, while each vertical domino covers squares whose sum is $-1$ mod 4. Hence, the total sum covered by $S$ is $2022 \equiv 2$ mod 4. However, each L-shape tetromino (which contains one horizontal and one vertical domino) covers squares who sum is 0 mod 4, a contradiction. Thus, at least one of the statements is false.

\begin{question}\label{A::2023-O-2-3}
    Let $n \geq 2$ be a positive integer. For a positive integer $a$, let $Q_a(x) = x^n + ax$. Let $p$ be a prime and let $S_a = \bc{b \mid 0 \leq b \leq p - 1, \exists c \in \ZZ, Q_a(c) \equiv b \pmod{p}}$. Show that $\frac{1}{p-1} \sum_{a=1}^{p-1} \abs{S_a}$ is an integer.
\end{question}

See \hyperref[A::2022-O-2-5]{2022 Open Round 2 Question 5}.

\begin{question}\label{A::2023-O-2-4}
    Find all functions $f : \ZZ \to \ZZ$, such that \[f(x+y)((f(x) - f(y))^2 + f(xy)) = f(x^3) + f(y^3)\] for all integers $x$, $y$.
\end{question}

\resqh{https://artofproblemsolving.com/community/c6h3102144p28048188}{AoPS thread}

\solctt{https://artofproblemsolving.com/community/c6h3102144p28049134}{Ld\_minh4354} Let $P(x, y)$ be the assertion that $f(x+y)((f(x) - f(y))^2 + f(xy)) = f(x^3) + f(y^3)$. From $P(x, x)$, one has \[f(2x)f(x^2) = 2f(x^3).\] $P(0, 0)$ hence gives us $f(0)^2 = 2f(0)$, whence $f(0) = 0$ or $f(0) = 2$. This gives us two main cases:

\case{1} Suppose $f(0) = 0$. Then $P(x, 0)$ gives us \[P(x, 0) : \qquad f(x)^3 = f(x^3).\] From $P(1, 1)$, we hence have \[P(1, 1) : \qquad f(2)f(1) = 0.\] Thus, $f(1) = 0$ or $f(2) = 0$.

\subcase{1A} Suppose $f(1) = 0$. $P(-1, 1)$ clearly gives $f(-1) = 0$. We now prove inductively that $f(x) \equiv 0$. The base case $x = 0$ has already been assumed. Now suppose that $f(k) = 0$ for some $k \in \ZZ$. Applying this inductive hypothesis to $P(k, 1)$ and $P(-k, 1)$, we immediately get $f(k + 1) = f(k - 1) = 0$. This closes the induction. One solution is hence $f(x) \equiv 0$.

\subcase{1B} Suppose $f(2) = 2$. Then we have the following:
\begin{alignat*}{2}
    P(x, 0) &: \qquad f(x)^3 &&= f(x^3)\\
    P(1, 0) &: \qquad f(1) &&\in \bc{-1, 0, 1}\\
    P(-1, 0) &: \qquad f(-1) &&\in \bc{-1, 0, 1}
\end{alignat*} 
If $f(1) = 0$, then by Subcase 1A, we would have $f(x) \equiv 0$, contradicting $f(2) = 2$. If $f(-1) = 0$, then by $P(-1, -1)$, we would have $f(-2)f(1) = 0$. However, if $f(-2) = 0$, then by $P(-2, 2)$, one gets $0 = 8$, a contradiction. Thus, $f(1), f(-1) \neq 0$.

From $P(1, -1)$, we have $f(1) + f(-1) = 0$. Suppose $f(1) = -1$. By $P(2, 1)$, we have $f(3) = \frac8{11} \notin \ZZ$, a contradiction. Thus, $f(1) = 1$ and $f(-1) = -1$. We now show that $f(x) \equiv x$ via induction. The base case has already been settled (namely $x = \bc{-1, 0, 1})$. Now suppose that $f(x) = x$ on $[-k, k]$ for some $k \in \ZZ$. From $P(k, 1)$ and $P(-k, 1)$, applying the inductive hypothesis yields
\begin{alignat*}{2}
    P(k, 1) &: \qquad f(k+1)(k^2 - k + 1) = k^3 + 1 &&\implies f(k+1) = k+1\\
    P(-k, 1) &: \qquad f(-k-1)(k^2 - k + 1) = -k^3 -1 &&\implies f(-k-1) = -k-1
\end{alignat*}
This closes the induction. We hence have a second solution, namely $f(x) \equiv x$.

\case{2} Suppose $f(0) = 2$. From $P(x, 0)$ one gets \[P(x, 0) : \qquad f(x)(f(x)^2 - 4f(x) + 6) = f(x^3) + 2. \tag{1}\] Taking $P(1, 0)$ hence gives us a cubic in $f(1)$: \[P(1, 0) : \qquad f(1)^3 - 4f(1)^2 + 5f(1) - 2 = 0.\] We thus have $f(1) = 1$ or $f(1) = 2$.

\subcase{2A} Suppose $f(1) = 1$. Taking $P(-1, 1)$, one has \[P(-1, 1): \qquad 2f(-1)^2 - 3f(-1) + 1 = 0,\] whence $f(-1) = 1$. Note that $f(-1)$ is an integer and hence cannot be $\frac12$. We now show via induction that $f(x) = 1$ when $x$ is odd, and $f(x) = 2$ when $x$ is even. The base cases (namely $x \in \bc{-1, 0, 1}$) have already been settled. Let $k$ be some integer. Suppose that $f(k) = 1$ when $k$ is odd and $f(k) = 2$ when $k$ is even. From $P(k, 1)$, we have
\[P(k, 1) : \qquad f(k + 1) = \frac{f(k^3) + 1}{f(k^2) - f(k) + 1},\] from which it follows that $f(k + 1) = 2$ when $k + 1$ is even, and $f(k + 1) = 1$ when $k + 1$ is odd. Also, from $P(k, -1)$, we have \[P(k, -1) : \qquad f(k-1) = \frac{f(k^3) + 1}{f(k)^2 - 2f(k) + f(-k) + 1},\] from which it follows that $f(k - 1) = 2$ when $k - 1$ is even, and $f(k - 1) = 1$ when $k - 1$ is odd. This closes the induction. We hence obtain a third solution, namely \[f(x) \equiv \begin{cases}
    1, &\text{$x$ odd}\\
    2, &\text{$x$ even}
\end{cases}.\]

\subcase{2B} Suppose $f(1) = 2$. From $P(1, -1)$, we obtain a quadratic in $f(-1)$: \[P(1, -1) : \qquad 2f(-1)^2 - 7f(-1) + 6 = 0,\] which has solutions 2 and $\frac32$. Thus, $f(-1) = 2$ (since $f(-1) \in \ZZ)$. We now show that $f(x) \equiv 2$ via induction. The base cases ($x \in \bc{-1, 0, 1}$) have already been settled. Now suppose $f(k) = 2$ for some $k \in \ZZ$. From $P(k, 1)$ and (1), we get \[P(k, 1) : \qquad f(k+1) = \frac{f(k)(f(k)^2 - 4f(k) + 6)}{f(k)^2 - 3f(k) + 4} = 2.\] Likewise, $P(k, -1)$ gives \[P(k, -1) : \qquad f(k-1) = \frac{f(k)(f(k)^2 - 4f(k) + 6)}{f(k)^2 - 3f(k) + 4} = 2.\] This closes the induction. We hence get our fourth and final solution: $f(x) \equiv 2$.

To conclude, the following four functions are the only solutions to the given functional equation: \[f(x) \equiv 0, \quad f(x) \equiv 2, \quad f(x) \equiv x, \quad f(x) \equiv \begin{cases}
    1, &\text{$x$ odd}\\
    2, &\text{$x$ even}
\end{cases}\]

\begin{question}[$x \in {(0, 120]}$]\label{A::2023-O-2-5}
    Determine all real numbers $x$ between 0 and 180 such that it is possible to partition an equilateral triangle into finitely many triangles, each of which has an angle of $x\deg$.
\end{question}

\resqh{https://artofproblemsolving.com/community/c6h3102149p28048211}{AoPS thread}

\solctt{https://artofproblemsolving.com/community/c6h3102149p28159955}{oneplusone} We claim that $x \in (0, 120]$. We split our proof into 3 cases:

\case{1} Suppose $x > 120$. Let there be a total of $n$ ``small triangles'' in the partition of the original equilateral triangle. Let $A$ be the set of vertices of the original equilateral triangle. Let $B$ be the set of vertices that lie on an edge. Let $C$ be the remaining vertices. Now observe that angle sum of all $n$ ``small triangles'' must be $180n\deg$. Since each vertex in $A$, $B$ and $C$ contributes $60\deg$, $180\deg$ and $360\deg$ respectively, we have the equality \[60\cdot3 + 180\abs{B} + 360\abs{C} = 180n \implies  \abs{B} + 2\abs{C} = n-1. \tag{1}\] Now observe that the vertices in $A$ cannot have an angle $x\deg$. On the hand, each vertex in $B$ and $C$ can have at least 1 and 2 such angles respectively. Thus, the total number of $x\deg$ angles is at least $n$ (by our assumption) and at most $\abs{B} + 2\abs{C}$. With (1), we get the contradiction $n \leq n-1$. Thus, $x > 120$ is impossible.

\case{2} Suppose $x = 120\deg$. This is clearly achievable; Let $O$ be the centre of the equilateral triangle $\triangle ABC$. Then $\triangle AOB$, $\triangle BOC$ and $\triangle COA$ all contain a $120\deg$ angle.

\case{3} Suppose $x < 120\deg$. For $R > 0$, let an $R$-trapezoid be a trapezoid similar to the trapezium $ABCD$, where $\angle A = \angle D = 60\deg$ and $\angle B = \angle C = 120\deg$, with $AB = CD = 1$ and $BC = R$. We call a shape constructible if it can be partitioned into triangles, each of which has an angle $x \deg$. Firstly, observe that for $R$ sufficiently large, the $R$-trapezoid is constructible. This is shown in the figure below:
\begin{center}
    \begin{tikzpicture}[scale=2.5]
        \coordinate[label=below right:$A$] (A) at (2, 0);
        \coordinate[label=below left:$B$] (B) at (-2, 0);
        \coordinate[label=above left:$C$] (C) at (-3/2, 0.866);
        \coordinate[label=above right:$D$] (D) at (3/2, 0.866);
        \coordinate (E) at (-1.7, 0);
        \coordinate (E') at (0.479, 0);
        \coordinate (C') at (0.279, 0.866);
        \coordinate (F) at (-0.611, 0.866);
        \coordinate (H) at (1.7, 0);
        
        \node at (-0.611, 0.6) {$x\deg$};

        \draw (A) -- (B) -- (C) -- (D) -- (A);
        \draw (C) -- (E) -- (F) -- (E');
        \draw (C') -- (H);

        \draw[->-=0.5] (E') -- (C');
        \draw[->-=0.5] (H) -- (D);

        \draw pic [draw, angle radius=3mm, ""] {angle = C--E--B};
        \draw pic [draw, angle radius=3mm, ""] {angle = E--C--F};
        \draw pic [draw, angle radius=3mm, ""] {angle = E--F--E'};
        \draw pic [draw, angle radius=3mm, ""] {angle = F--C'--E'};
        \draw pic [draw, angle radius=3mm, ""] {angle = H--E'--C'};
        \draw pic [draw, angle radius=3mm, ""] {angle = C'--D--H};
        \draw pic [draw, angle radius=3mm, ""] {angle = A--H--D};

        \draw[<->] (0.279, 0.9) -- (3/2, 0.9);
        \node[anchor=south] at (0.89, 0.9) {arbitrarily long};

    \end{tikzpicture}
\end{center}
It thus follows that the 1-trapezoid is constructible: simply slice it horizontally into sufficiently thin $R$-trapezoids (for sufficiently large $R$). Since an equilateral triangle can be partitioned into 3 1-trapezoids, it must also be constructible.

Thus, $x \in (0, 120]$.
