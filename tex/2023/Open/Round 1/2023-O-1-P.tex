\subsection{Round 1 Problems}
Solutions can be found in Section~\ref{S::2023-O-1}.

\begin{enumerate}
    \hyperrefitem[Q::2023-O-1-1] The graph $C$ with equation $y = \dfrac{ax^2 + bx + c}{x+2}$ has an oblique asymptote with equation $y = 4x - 6$ and one of the stationary points at $x = -4$. Find the value of $a + b + c$.
    \hyperrefitem[Q::2023-O-1-2] If $x = \dfrac{1}{1} + \dfrac{1}{1+2} + \dfrac{1}{1+2+3} + \dfrac{1}{1+2+3+4} + \cdots + \dfrac{1}{1 + 2 + 3 + \cdots + 100}$, find the value of $\floor{1010x}$.
    \hyperrefitem[Q::2023-O-1-3] The set of all possible values of $x$ for which the sum of the infinite series \[1 + \dfrac16 \bp{x^2 - 5x} + \dfrac1{6^2} \bp{x^2 - 5x}^2 + \dfrac1{6^3} \bp{x^2 - 5x}^3 + \cdots\] exists can be expressed as $(a, b) \cup (c, d)$, where $a < b < c < d$. Find $d - a$.
    \hyperrefitem[Q::2023-O-1-4] Find the value of $\floor{y}$, where $y = \displaystyle\sum_{k=0}^\infty (2k+1)(0.5)^{2k}$.

    \textit{(Hint: Consider the series expansion of $(1 - x)^{-2}$)}
    \hyperrefitem[Q::2023-O-1-5] The solution of the inequality $\abs{x-1} + \abs{x+1} < ax + b$ is $-1 < x < 2$. Find the value of $\floor{a + b}$.
    \hyperrefitem[Q::2023-O-1-6] The equation $x^4 - 4x^2 + qx - r = 0$ has three equal roots. Find the value of $\floor{\dfrac{3q^2}{r^2}}$.
    \hyperrefitem[Q::2023-O-1-7] The parabolas $y = x^2 - 16x + 50$ and $x = y^2$ intersect at 4 distinct points which lie on a circle centred at $(a, b)$. Find $\abs{a - b}$.
    \hyperrefitem[Q::2023-O-1-8] In the 3-dimensional Euclidean space with origin $O$ and three mutually perpendicular $x$-, $y$- and $z$-axes, two planes $x + y + 3z = 4$ and $2x - z = 6$ intersect at the line $\vec r \crossp \cveciii{-1}{a}{b} = \cveciii{-2}{c}{d}$. Find the value of $\abs{a + b + c + d}$.
    \hyperrefitem[Q::2023-O-1-9] Let $x$, $y$, $z$ be real numbers with $3x + 4y + 5z = 100$. Find the minimum value of $x^2 + y^2 + z^2$.
    \hyperrefitem[Q::2023-O-1-10] Find the area of the region represented by the equation $\floor{x} + \floor{y} = 1$ in the region $0 \leq x < 2$.

    \textit{(Note: If you think that there is no area defined by the graph, enter ``0''; if you think that the area is infinite, enter ``9999''.)}
    \hyperrefitem[Q::2023-O-1-11] Let $ABC$ be a triangle satisfying the following conditions that $\angle A + \angle C = 2\angle B$, and $\dfrac1{\cos A} + \dfrac1{\cos C} = \dfrac{-\sqrt2}{\cos B}$. Determine the value of $\dfrac{2022\cos{\frac{A-C}{2}}}{\sqrt2}$.
    \hyperrefitem[Q::2023-O-1-12] Find $x$ which satisfies the following equation \[\dfrac{x-2019}{1} + \dfrac{x-2018}{2} + \dfrac{x-2017}{3} + \cdots + \dfrac{x+2}{2022} + \dfrac{x+3}{2023} = 2023.\]
    \hyperrefitem[Q::2023-O-1-13] Assume that $x$ is a positive number such that $x - \frac1x = 3$ and \[\dfrac{x^{10} + x^8 + x^2 + 1}{x^{10} + x^6 + x^4 + 1} = \dfrac{m}{n},\] where $m$ and $n$ are positive integers without common factors larger than 1. Determine the value of $m + n$.
    \hyperrefitem[Q::2023-O-1-14] Consider the set of all possible pairs $(x, y)$ of real numbers that satisfy $(x-4)^2 + (y-3)^2 = 9$. If $S$ is the largest possible value of $\dfrac{y}{x}$, find the value of $\floor{7S}$.
    \hyperrefitem[Q::2023-O-1-15] Let $x$, $y$ be positive integers with $16x^2 + y^2 + 7xy \leq 2023$. Find the maximum value of $4x + y$.
    \hyperrefitem[Q::2023-O-1-16] Let $x$ be the largest real number such that \[\sqrt{x - \dfrac1x} + \sqrt{1 - \dfrac1x} = x.\] Determine the value of $(2x-1)^4$.
    \hyperrefitem[Q::2023-O-1-17] Two positive integers $m$ and $n$ differ by 10 and the digits in the decimal representation of $mn$ are all equal to 9. Determine $m + n$.
    \hyperrefitem[Q::2023-O-1-18] Let $\bc{a_n}$ be a sequence of positive numbers, and let $S_n = a_1 + a_2 + a_3 + \cdots + a_n$. For any positive integer $n$, let $b_n = \dfrac12 \bp{\dfrac{a_{n+1}}{a_n} + \dfrac{a_n}{a_{n+1}}}$. Given that $\dfrac{a_n + 2}{2} = \sqrt{2S_n}$ holds for all positive integers $n$, determine the limit $\lim_{n \to \infty} (b_1 + b_2 + \cdots + b_n - n)$.
    \hyperrefitem[Q::2023-O-1-19] Let $ABC$ be a triangle with $AB = c$, $AC = b$ and $BC = a$, and satisfies the conditions $\tan C = \dfrac{\sin A + \sin B}{\cos A + \cos B}$, $\sin{B - A} = \cos C$ and that the area of triangle $ABC = 3 + \sqrt{3}$. Determine the value of $a^2 + c^2$.
    \hyperrefitem[Q::2023-O-1-20] \textbf{[VOID]} Let $g \colon \RR \to \RR$, $g(0) = 4$ and that \[g(xy + 1) = g(x)g(y) - g(y) - x + 2023.\] Find the value of $g(2023)$. 
    \hyperrefitem[Q::2023-O-1-21] In the triangle $ABC$, $D$ is the midpoint of $AC$, $E$ is the midpoint of $BD$, and the lines $BA$ and $CE$ are tangent to the circumcircle of the triangle $ADE$ at $A$ and $E$ respectively. Suppose the circumradius of the triangle $AED$ is $(\frac{64}{7})^{1/4}$. Find the area of the triangle $ABC$.
    \hyperrefitem[Q::2023-O-1-22] $ABCD$ is a parallelogram such that $\angle ABC < 90\deg$ and $\sin \angle ABC = \frac45$. The point $K$ is on the extension of $BC$ such that $DC = DK$; the point $L$ is on the extension of $DC$ such that $BC = BL$. The bisector of $\angle CDK$ intersects the bisector of $\angle LBC$ at $Q$. Suppose the circumradius of the triangle $ABD$ is 25. Find the length of $KL$.
    \hyperrefitem[Q::2023-O-1-23] A group of 200 monkeys is given the task of picking up all 3000 peanuts on the ground. Determine the maximum number $k$ such that there must be $k$ monkeys picking up the same number of peanuts. (It is possible that some lazy monkeys may not pick up any peanuts at all).
    \hyperrefitem[Q::2023-O-1-24] A chain of $n$ identical circles $C_1, C_2, \ldots, C_n$ of equal radii and centres on the $x$-axis lie inside the ellipse $E: \frac{x^2}{2023} + \frac{y^2}{333} = 1$ such that $C_1$ is tangent to $E$ internally at $(-\sqrt{2023}, 0)$, $C_n$ is tangent to $E$ internally at $(\sqrt{2023}, 0)$, and $C_i$ is tangent to $C_{i+1}$ externally for $i = 1, \ldots , n-1$. Determine the smallest possible value of $n$.
    \hyperrefitem[Q::2023-O-1-25] Let $p > 2023$ be a prime. Determine the number of positive integers $n$ such that $(n-p)^2 + 2023(2023 - 2n - 2p)$ is a perfect square.
\end{enumerate}
